\chapter{The Text Editor}

Nuprl text consists of sequences of characters and {\em definition
references}.  Since these sequences are stored internally as recursive
trees, they are also known as {\em text{} trees}, or {\em T-trees}{}.
The text{} editor, {\em ted}, is a structure{} editor; the cursor on
the screen represents a cursor into the text tree being edited, and
movements of the tree cursor are mapped into movements of the screen's
cursor.

The text editor is used to create and modify DEF, ML and EVAL objects
as well as rules and main goals of theorems.
Changes to objects are reflected immediately.

The text editor is almost completely modeless.
To insert a character, type it;
to insert a definition, {\em instantiate} it;
to remove a character, delete it or \KILL{} it;
to move the cursor, use the arrow keys;
to exit the editor, use \EXIT.
There is a {\em bracket} mode, explained in the next section,
which changes the way definition references are displayed.

\section{Tree Cursors, Tree Positions and Bracket Mode}

A {\em tree{} cursor}{}
is a path from the root of a text tree to a
{\em tree{} position}, namely
a character, a definition reference or the end of a subtree.
When text is being edited the tree cursor is mapped to a location on the
screen,
and this is where the editor places the screen cursor.
Unfortunately, more than one tree position can map to a screen
position.
Consider the following definition of the $\geq$ relation.
\begin{verbatim}
        <x>>=<y>==((<x><<y>) -> void).
\end{verbatim}
In the tree represented by {\tt 3>=1} there are six tree positions:
the entire definition,
the character {\tt 3},
the end of the subtree containing {\tt 3},
the character {\tt 1},
the end of the subtree containing {\tt 1} and
the end of the entire tree.
These six tree positions are mapped into four screen positions:
\begin{verbatim}
        3>=1
        ^^ ^^
\end{verbatim}
The first is for both the entire definition and the character {\tt 3},
the second is for the end of the subtree containing {\tt 3},
the third is for the character {\tt 1},
and the fourth is for the end of the subtree containing {\tt 1} and also for
the end of the entire tree.
If the screen cursor were under the {\tt 3}, pressing the \KILL{} key would
either delete the entire definition or just the {\tt 3}, depending on where the
{\em tree} cursor actually was.

To eliminate this ambiguity the window may be put into
{\em bracket{} mode\/};
in this mode all definition references are surrounded by square brackets, and the screen
position that corresponds to the entire definition reference is the left square bracket:
\begin{verbatim}
        [3>=1]
        ^^^ ^^^
\end{verbatim}
Notice that the right square bracket corresponds to the end of the subtree
containing {\tt 1}; this gets rid of the other ambiguous screen position.

For another example of an ambiguity removed by bracket mode, consider the
two trees
\begin{verbatim}
        [23>=1]   and   2[3>=1].
\end{verbatim}
The first is quite reasonable, but the second is probably a mistake, since
it expands to {\tt 2((3<1) -> void)}.
In unbracketed mode both trees would be displayed as {\tt 23>=1}.

The bracket mode toggle, \MODE, switches the current window into or
out of bracket mode.

\section{Moving the Cursor}{}

The cursor can be cycled through the edit windows with the $\JUMP$ key.
This is most useful when copying text from one window to another.
To move the cursor around within the current window, use the arrow and
\LONG{} keys, as follows.

\begin {description}
\item[\UP, \DOWN]{} \hfill \break
\noindent
    Move the screen cursor up or down one line.
    The tree cursor is moved to the tree position that most closely corresponds
    to the screen position one line up or down.
    If the new tree position is not already being shown in the text window,
    scroll the window.

\item[\LEFT, \RIGHT]{} \hfill \break
\noindent
    Move the tree cursor left or right one position.
    That is, move one position backward or forward in a preorder walk of the
    tree.
    Keep in mind that, as explained above,
    this one-position change of the tree cursor does not
    always correspond to a one-column change of the screen cursor.  

\item[\LONG{} \UP, \LONG{} \DOWN]{} \hfill \break
\noindent
    Move up or down four lines in the screen.
    If necessary, scroll the window.

\item[\LONG{} \LEFT, \LONG{} \RIGHT]{} \hfill \break
\noindent
    Move the tree cursor left or right four positions.

\item[\LONG{} \LONG{} \UP, \LONG{} \LONG{} \DOWN]{} \hfill \break
\noindent
    Move up or down a screen.

\item[\LONG{} \LONG{} \LEFT, \LONG{} \LONG{} \RIGHT]{} \hfill \break
\noindent
    Move the screen cursor to the left or right end of the current line.
    The tree cursor is moved left or right to match.
\end {description}

If \LONG{} is used before any other key besides an arrow key it is ignored.
(The \LONG{} \DIAG{} command works only in the proof editor.)
Also, any extra \LONG{}s are ignored.

The cursor may also be moved within a single window by pointing at the
desired location and clicking mouse-\JUMP.  However, note that scrolling
the contents of the window cannot be done just with the mouse.  Pointing at
a non-active window and clicking mouse-\JUMP\ will make the window active
(but leave the cursor where it was when the window was last active).

\section{Adding Text}

{}To add a character, simply type it; it will be added to the left of the
current position.
This also includes the newline character; to start a new line, enter
carriage return.

To {\em instantiate}{} a definition, use \INS.  Nuprl will prompt for the name of
the definition in the command window with {\tt I$\,$>}; type the name of the definition
and press \CR.  (The name can be abbreviated in the same way as in command
arguments---see above.)  Nuprl will insert an instance of the definition to the
left of the current position and move the tree cursor to the first
parameter.

\section{Deleting and Killing Text}

There are two ways to remove a character.
The \KILL\ key removes the one at the current position,
while
the $\DEL${} key removes the one to the left of the current position,
assuming it is in the same subtree as the current position.
(If it is not \DEL{} does nothing.)
To remove a definition reference, use $\KILL${}.{}
The cursor{} must point to the beginning of the reference.
A {\em selection}{} of text may also be killed.
This involves three steps.
\begin {enumerate}
\item With the $\SEL$ key choose the left or right endpoint of
    the selection.  Alternatively, point at the endpoint with the
    mouse and click mouse-\SEL\ (this will not move the cursor).

\item Select the other endpoint.
    It must be in the same window as the first.
    Also, if the first endpoint was within an argument to a definition,
    the second must be within the same argument.

\item Press \KILL; \prl{} will delete the two endpoints and everything between.
\end {enumerate}

The killed text is moved to the {\em kill{} buffer}
and can be used by the \COPY{} command (see below).
Only killed text goes into the kill buffer; text deleted by \DEL{} does not.

There can be at most two selected endpoints at any time.
If another endpoint is selected the oldest one is forgotten.
When a \KILL{} or \COPY{} is done {\em both} endpoints are forgotten.
The two selected endpoints must be in the same window;
if they are not \KILL{} and \COPY{} will give error messages.

\section{Copying and Moving Text}

To{} copy text from one location to another, select it,
move the cursor to the
target location, and press the \COPY{} key.
If two endpoints were specified the selected text, including the endpoints,
is copied to the left of the current position.
If only one endpoint was specified the selected character or definition reference is
copied to the left of the current position.
In either case all endpoints are forgotten.

To move{} text, use \KILL{} and \COPY{} together;
if \COPY{} is pressed when no selections have been made the
kill buffer is copied to the left of the current position.
Therefore, to move some text, select it, \KILL{} it, move the cursor
to the target position, and press \COPY.
The kill buffer is not affected by a \COPY,
so the easiest way to replicate a piece of text in many places is to
type it, \KILL{} it and then use \COPY{} many times.

\section{Miscellaneous}

When typing at the command window, a mouse-\SEL\ while it is over a
definition instance in another ted (text editing) window will cause the
name of the definition to be inserted into the command window's input line.
So if you are reading and do not know what a certain notation means, you
can enter \CMD{} to get a {\tt P>} prompt, type ``view'', and then point at
the notation in question.  The name of the definition will be typed in for
you and a return will bring it up in a view window.  






