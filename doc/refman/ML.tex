\chapter{The Metalanguage}

\label{ML}

The programming language ML is a metalanguage for Nuprl's type
theory.  Its main use is to write programs called {\em tactics} that
automate the proof construction process.  

ML is not documented here.  The version of ML used in Nuprl is a minor
variant of
the ML used in the Edinburgh LCF system.  This is documented in the book
{\em Edinburgh LCF: A Mechanized Logic of Computation}, M.\ Gordon, R.\
Milner and C.\ Wadsworth, Springer-Verlag LNCS 78, 1979.  The language is
partially documented in the Nuprl book, which also gives some examples of
tactics.

In this section we describe the embedding of Nuprl's logical apparatus
within ML, and the structure of the tactic mechanism.  Many of the
details, such as ML primitives corresponding to the rules of the type
theory, and various utilities for accessing libraries and printing, are
left to appendices.

In what follows, as well as in Appendix~\ref{MLExts},
a description of an ML 
function will usually begin with an application of the
function to a list of typed
formal arguments.  For example, the {\tt max} function might be presented
as
\begin{quote}
{\tt max(x: int, y:int): int}.  \rm The maximum of \tid{x} and \tid{y}.
\end{quote}
This indicates that the type of max is {\tt int->int->int}. 

\section{Nuprl ML versus LCF ML}

The ML used by Nuprl has been adapted from the ML of LCF.  The changes are
as follows.
\begin{itemize}

\item All components related to LCF's logic have been removed, including
the term-quotation facility.  

\item New primitive functions and datatypes have been added (these are
documented below).  

\item New facilities for naming constants have been added.  In particular,
single forward quotes (``\tid{'}'') are used to delimit the text of Nuprl
terms, and a string enclosed in double backquotes (``\tid{`}'') is
interpreted as the list of tokens obtained by breaking the string into
words according to the occurrences of blanks.

\item The names of the functions mapping between an abstract type and its
implementation type have an underscore added.

\end{itemize}


\section{Basic Types in ML for Nuprl}

In specializing the ML language to Nuprl
we have made each of the {\em kinds} of
objects that occur in the logic into ML types.   These types are abstract
in the sense of the abstract type mechanism of ML, except that they are
defined in the implementation of ML.  Objects of the types can be created
or accessed only through the provided functions.  

There are four such types: {\tt term}, \tid{declaration}, \tid{rule} and
\tid{proof}.  The polymorphic equality test \tid{=} of ML has been extended
to these types in the following way.  Two objects of type {\tt term} are
equal if and only if they are $\alpha$-convertible variants of each other.
That is, they are equal if and only if the bound variables can be renamed
so that the terms are identical.  Unfortunately, equality on proofs does
not respect term equality.  Two objects of type {\tt proof} are equal if
they are {\em identical}.  In particular, two proofs which differ only
because of renaming of some bound variables will {\em not} be equal in ML.
Two objects of type {\tt rule} (or {\tt declaration}) are equal if and only
if they are identical.

\subsection*{\tt term}

The type {\tt term} is the type of terms of Nuprl's logic.

\manentry{term\_kind(t:term): tok}   A token naming the kind of term
\tid{t} is.  For example, if \tid{t} is the term $2+3$ then the result is
the token \tid{`ADDITION`}.  Possible results are: {\tt UNIVERSE},
{\tt VOID}, {\tt ANY}, {\tt ATOM}, {\tt TOKEN}, {\tt INT}, {\tt
NATURAL-NUMBER}, {\tt MINUS}, {\tt ADDITION}, {\tt SUBTRACTION},
{\tt MULTIPLICATION}, {\tt DIVISION}, {\tt MODULO}, 
{\tt INTEGER-INDUCTION}, {\tt LIST}, {\tt NIL}, {\tt CONS}, 
{\tt LIST-INDUCTION}, {\tt UNION}, {\tt INL}, {\tt INR}, {\tt DECIDE}, {\tt
PRODUCT}, {\tt PAIR}, {\tt SPREAD}, {\tt FUNCTION}, {\tt LAMBDA}, {\tt
APPLY}, {\tt QUOTIENT}, {\tt SET}, {\tt EQUAL}, {\tt AXIOM}, {\tt
VAR}, {\tt BOUND-ID}, {\tt TERM-OF-THEOREM}, {\tt
ATOMEQ}, {\tt INTEQ}, {\tt INTLESS}, and \tid{TAGGED}.

\smallskip

Each kind of term has associated with it three functions: a recognizer, a
destructor, and a constructor.  For example, for addition terms we have the
following.

\manentry{is\_addition\_term(t:term): bool}  

\manentry{destruct\_addition\_term(t:term): term\#term}  Fails if \tid{t}
is not an addition term.

\manentry{make\_addition\_term(a:term, b:term): term}

\smallskip

The analogous functions for the other term kinds are documented in
Appendix~\ref{MLExts}.  A constructor will fail if the result would not be
a term, such as if \tid{make\_universe\_term} were applied to a negative number.
Nuprl definitions appearing in a term are invisible to ML.

\note{Evaluating term-of}

In addition to the individual term constructors, term quotes may be used to
specify terms in ML.  For example, the ML expression 
\begin{quote}
\tt '2+3'
\end{quote}
is of type \tid{term} and has as its value the term $2+3$.  
Note that these quotation marks are
different from those used for tokens in ML.  {\em Warning.}  Term quotes
can {\em not} be used in ML {\em files}.  They can only be used in ML
library objects or in ML mode.  


    
\subsection*{\tt rule}

This is the type of (names of) \prl\ refinement rules.  
The constructors for this type do not correspond
precisely to the rules in \nuprl.  Refinement rules in \prl\ are usually
entered in the proof editor
as ``{\tt intro}'', ``{\tt elim}'', ``{\tt hyp}'', etc.  Strictly
speaking, the notation ``{\tt intro}'' refers not to a single refinement
rule but to a collection of introduction refinement rules, and the context
of the proof is used to disambiguate the intended introduction rule at the
time the rule is applied to a sequent.  There is a similar ambiguity with
the other {\em names} of the refinement rules.  In addition to this
ambiguity, the various sorts of the rules require different additional
arguments.  Because rules in ML may exist independently of the proof
context that allows the particular kind of rule to be determined, and
because functions in ML are required to have a fixed number of arguments,
the rule constructors have been subdivided beyond the ambiguous classes
that are normally visible to the \prl\ user.  For example, there are
constructors like {\tt universe\_intro\_integer}, {\tt product\_intro} and
{\tt function\_intro}, all of which are normally designated in proof
editing with {\tt intro}.  
These rule constructors are given in Appendix~\ref{MLExts}.  

There is also a generic constructor that can be used when the 
proof context is known.

\manentry{parse\_rule\_in\_context(x:tok, p:proof):rule}  
This function uses the same
parser that is used when a rule is typed into the proof editor.  

\smallskip

There are no destructors, but there are several marginally useful functions
which do a limited amount of rule analysis.  The first two are used in the
transormation tactic \tid{copy}.

\manentry{get\_assum\_number(r:rule): int}  If \tid{r} is a rule which refers to
a single hypothesis, return the number of the hypothesis.

\manentry{set\_assum\_number(r:rule, i:int): rule}  If \tid{r} is as above, modify
the rule to refer to the hypothesis numbered \tid{i}.

\manentry{rule\_kind(r:rule): tok} The ``kind'' of \tid{r}.  At present
this is the internal name of the rule, which is not necessarily the same as
what appears in the name of the corresponding rule constructor.  The actual
names are not documented here.


\subsection*{\tt declaration}

This is the type of \prl\ bindings that occur in proofs.  A declaration
associates a variable with a type (term).  The declarations in a \prl\
sequent occur on the left of the {\tt >>}.


\manentry{construct\_declaration(x:tok, T:term): declaration}  
A declaration where the
variable \tid{x} is associated with the type \tid{T}.

\manentry{destruct\_declaration(d:declaration): tok\#term}

\manentry{id\_of\_declaration(d:declaration): tok}

\manentry{type\_of\_declaration(d:declaration): term}

\subsection*{\tt proof}

This is the type of partial \nuprl\ proofs.  Proofs have the structure
described in Section~\ref{Overview}.  The related constructors, destructors
and predicates are given below.  The fact that (non-degenerate) proofs can
only be built via the refinement functions guarantees that all objects of
type \tid{proof} represent proofs.

\manentry{hypotheses(p:proof): declaration list} 
The hypotheses of the root sequent of the proof \tid{p}. 

\manentry{conclusion(p:proof): term} 
The term to the right of the turnstile (\tid{>>}) in the root sequent
of \tid{p}.

\manentry{refinement(p:proof): rule}  The rule at the root of \tid{p}, if
one is present (otherwise fail).

\manentry{children(p:proof): proof list}  The children of the root of
\tid{p}.  The function fails if there is no refinement.

\manentry{refine(r:rule): tactic}  The type \tid{tactic} is described in the
next section.  It is defined as the type
\begin{quote}
\tt proof -> proof list \# (proof list -> proof).
\end{quote}
The function {\tt refine(r)} when 
applied to a proof $p$ applies the rule \tid{r}
to the sequent $s$ at the root of $p$.  This either generates a list $l$ of
subgoals according to the specifications given in Chapter~\ref{Rules}, or,
in the case the rule is inapplicable, fails.  The result of \tid{refine(r)(p)}
is the pair $(l,f)$ where $f$ is a function (called a {\em validation})
that takes a list \id{pl} of proofs, checks whether the list of root
sequents is equal to $l$ (failing if not), and if so creates a new proof
whose root sequent is the root sequent of $p$, whose refinement rule is
$r$, and whose children are \id{pl}.

The composition of \tid{refine} with \tid{parse\_rule\_in\_context} (see
above) can be used to simulate the proof editor, where a string is
interpreted as a specification of a rule.


\section{Tactics}





The type of tactics is
\begin{verbatim}
    tactic = proof -> proof list # (proof list -> proof),
\end{verbatim}
The component of a tactic that has type \tid{proof list -> proof} is
called the {\em validation}.  The basic idea is that a tactic decomposes a
goal proof $g$ into a finite (possibly empty) list of subgoals, $g_1,\ldots
,g_n$, and also produces a validation $v$ which ``jusfifies'' the reduction
of the goals to the subgoals.  The subgoals are degenerate proofs, i.e.,
proofs which have only the top-level sequent filled in and have no
refinement rule or subgoals.  We say that a proof $p_i$ {\em achieves} the
subgoal $g_i$ if the top-level sequent of $p_i$ is the same as the
top-level sequent of $g_i$.  The validation, $v$, is supposed to take a
list $p_1,\ldots ,p_n$, where each $p_i$ achieves $g_i$, and produce a
proof $p$ that achieves $g$. (Note that wherever we speak of proof in the
context of ML, we mean possibly incomplete proofs.)  Because of the
combinator language for tactic construction it is rarely necessary to
explicitly refer to validations.


\subsection*{Tacticals}

Tacticals allow existing tactics
to be combined to form new tactics.


\begin{description}

\item[{\tt IDTAC:}]
{\tt IDTAC}\index{\IDTAC\ tactic}{}
is the identity\index{identity tactic}{} tactic.  The result of
applying {\tt IDTAC} to a
proof is one subgoal, the original proof, and a validation that, when
applied to an achievement of the proof, returns the achievement.  

\item [{\tt $tac_1$ THEN $tac_2$:}]
The {\tt THEN}\index{\THEN\ tactical}{}
tactical is the composition functional for tactics.
{\tt $tac_1$ THEN $tac_2$} defines a tactic
which when invoked on a proof first applies $tac_1$ and then, to each
subgoal produced by $tac_1$, applies $tac_2$.

\item [{\tt $tac$ THENL $tac\_list$:}] The {\tt THENL}\index{\THENL\ tactical}{}
tactical is similar to the
{\tt THEN} tactical except that the second argument is a list of tactics
rather than just one tactic.  The resulting tactic applies $tac$, and
then to each of the subgoals (a list of proofs) it applies the corresponding
tactic from $tac\_list$.  If the number of subgoals is different from
the number of tactics in $tac\_list$, the tactic fails.


\item [{\tt $tac_1$ ORELSE $tac_2$:}]  The
{\tt ORELSE}\index{\ORELSE\ tactical}{} tactical creates a tactic
which when applied to a proof first applies $tac_1$.  If this application
fails, or fails to make progress,
then the result of the tactic is the application of $tac_2$ to the
proof.  Otherwise it is the result that was returned by $tac_1$.  This
tactical gives a simple mechanism for handling failure of tactics.

\item [{\tt REPEAT $tac$:}]   The {\tt REPEAT}\index{\REPEAT\ tactical}{}
tactical  will repeatedly apply
a tactic until the tactic fails.  That is, the tactic is applied to the goal
of the argument proof, and then to the subgoals produced by the tactic, and
so on.  The {\tt REPEAT} tactical will catch all failures of the argument
tactic and can not generate a failure.


\item [{\tt COMPLETE $tac$:}]  The {\tt COMPLETE}\index{\COMPLETE\ tactical}{}
tactical constructs a tactic
which operates exactly like the argument tactic except that the new tactic
will fail unless a {\em complete} proof was constructed by $tac$, i.e.,
the subgoal list is null.  

\item [{\tt PROGRESS $tac$:}] The {\tt PROGRESS}\index{\PROGRESS\ tactical}{}
tactical constructs a tactic
which operates exactly like the argument tactic except that it fails unless
the tactic when applied to a proof makes some progress toward a proof.  In
particular, it will fail if the subgoal list is the singleton list
containing exactly the original proof.  

\item [{\tt TRY $tac$:}]  The {\tt TRY}\index{\TRY\ tactical}{}
tactical constructs a tactic which
tries to apply $tac$ to a proof; if this fails it returns the result
of applying {\tt IDTAC} to the proof.  This insulates the context in which
{\em tac} is being used from failures.  This is often
useful for the second argument of an application of the {\tt THEN}
tactical.

\end{description}




\section{Print Representation}

There are two kinds of functions to convert Nuprl syntactic objects
into string representations.  One converts objects to ML tokens, and the
other builds a representation in a file.

\manentry{term\_to\_tok(t:term): tok}  A string similar to what would
appear in a Nuprl window.  Note that this may include the display forms of
Nuprl definitions, and so there is no inverse function to recover \tid{t}
from the string.  This remark applies also to the following two functions.

\manentry{rule\_to\_tok(r:rule): tok}

\manentry{declaration\_to\_tok(d:declaration): tok}

\manentry{print\_term(t:term): void}
Print \tid{t} in the command window.  This works only in ML mode.

\manentry{print\_rule(r:rule): void}
Print a representation of \tid{r} in the command window.

\manentry{print\_declaration(d:declaration): void}



\manentry{latexize\_file(input\_file: tok; output\_file: tok): void} Create
a source file for Latex.  The \tid{input\_file} must be the result of
\tid{mm}, \tid{ppp}, \tid{print\_library} or a ``snapshot'' (\PRINT).  The
flag described next must have been set appropriately.

\manentry{set\_output\_for\_latexize(flag: bool) : void} This sets a flag
which may be ignored if ``special'' characters are not used.  (Special
characters are the ones outside the standard character set that require
Nuprl's fonts in order to be printed.)  If the output from \tid{mm},
\tid{ppp}, \tid{print\_library} or a snapshot command (see below) is to be
run through latexize\_file then the flag should be set to \id{true}.  If
the output is to be treated as standard ascii text then the flag should be
set to false.  The difference in the two kinds of output is in how special
characters are treated.  In the first case they are written out using a
code (usually a control character); \tid{latex\_file} will convert these to
the appropriate Latex commands.  In the second case, a directly-readable
character-string representation is written out.  Initially, the flag is
true.

\manentry{print\_library(first:tok, last:tok, file\_name: tok): void}
Print the
following components of the segment of the current library between
objects first and last to the named file:
\begin{enumerate}
\item the name and status of
every library object,
\item the body of every DEF, ML, and EVAL object, and
\item the extraction (if it currently exists) of any theorem whose name
ends with an underscore.  
\end{enumerate} 
The tokens \tid{`first`} and \tid{`last`} may be given as
the arguments \tid{first} and \tid{last} and are interpreted to refer to
the first and last library objects, respectively.

\manentry{mm: tactic}
Print a proof to a file.  The next function works similarly to \tid{mm} but
produces better looking output.
\tid{mm} is the same as IDTAC
except that as a side effect it prints out the nodes of its proof
argument in the order visited by a depth first traversal of the proof
tree.  The output is appended to whatever file snapshot output would go
to.  Indenting is used to indicate depth, and redundant hypotheses are
not displayed.  Vertical lines (or an approximation thereof) are printed
to help the reader keep track of levels, and tick marks (hyphens,
actually) are put on the lines to indicate the beginning of a conclusion
of a proof node.  Given an arbitrary turnstyle (``\tid{>>}'') in the printout,
to the right of it will appear the conclusion of a node of the proof
tree.  Directly above will be a numbered list of the hypotheses of the
proof node which are different from the hypotheses of the nodes parent
node.  The parent is found by following upward the first vertical line
to the left of the turnstyle.  Right after the conclusion of the node is
the refinement rule, and below that, indented once from the turnstyle,
are the children of the node.  ``INCOMPLETE'' in place of a rule indicates
that no rule was applied there.

\manentry{ppp: tactic}
Pretty-print a Nuprl proof tree. 

Print format for proof node is schematically as follows.
\begin{verbatim}
[...branches....]| 1. [id. (if not NIL)][...Hypothesis term.1........]
[...branches....]|    [............Hypothesis term.1.................]
       .                   .                             .
       .                   .                             .
[...branches....]|    [............Hypothesis term.1.................]
       .                   .                             .
       .                   .                             .
       .                   .                             .
[...branches....]| n. [id. (if not NIL)][...Hypothesis term.n........]
[...branches....]|    [............Hypothesis term.n.................]
       .                   .                             .
       .                   .                             .
[...branches....]|    [............Hypothesis term.n.................]
[...branches....]|->> [............Conclusion term...................]
[...branches....]?    [............Conclusion term...................]
       .                   .                             .
       .                   .                             .
[...branches....]?    [............Conclusion term...................]
[...branches....]? BY [............Refinement rule...................]
[...branches....]? !  [............Refinement rule...................]
       .                   .                             .
       .                   .                             .
[...branches....]? !  [............Refinement rule...................]
[...branches....]? !  
\end{verbatim}
where
\begin{enumerate}
\item Branch at ?'s printed only if node is not last sibling.
\item branch at !'s printed only if node has children.
\item term and rule text is folded so as to always stay in the appropriate
   region. 
\item A hypothesis is only printed if it is textually different from the
   same numbered hypothesis in the parent node.
\item If a hypothesis is hidden, []'s are put around the hypothesis number.
\item The root node does not have a branch entering the node.
\end{enumerate}

\manentry{(file\_name:tok): tactic} Set snapshot file to \tid{file\_name}
then apply  \tid{ppp}.

\manentry{set\_pp\_width(i:int): void} Set the number of columns to print
the proof with (default 105).

\manentry{complete\_nuprl\_path(directories: tok list; file\_name: tok):
tok} Make a complete pathname from the current ``nuprl path prefix'',
\tid{directories}, and \tid{file\_name}.  The result is formed by
concatenating the path prefix, the strings in the list \tid{directories}
and \tid{file\_name}, separating with the string ``\tid{/}'' (on Unix-like
hosts).  The main use of this function is in the ``load-all'' files in the
main library directory of Nuprl it provides a relatively host independent
way to specify the loading of other ML files.  If Nuprl has been installed
properly, then the default path prefix will be the root directory of Nuprl.

\manentry{set\_nuprl\_path\_prefix(pathname: tok): void} Set Nuprl's
path prefix.  

\manentry{set\_snapshot\_file(file\_name:tok): void} Directs subsequent
output of the snapshot feature (\PRINT) to the named file.




\note{Mark and Copy}


\section{Further Primitives for Terms and Proofs}

Most of the ML functions in this section could have been implemented in ML.
They are primitive either for efficiency reasons, or because a Lisp
implementation (Lisp is the implementation language of both Nuprl and ML)
already existed.  Many of the functions produce values that depend on the
current state of Nuprl's library.

\subsection*{For Terms}

\manentry{map\_on\_subterms(f: term -> term; t: term): term}
  Replace each immediate
subterm of t by its value under f.

\manentry{subterms(t: term): term list}
  The immediate subterms of t.

\manentry{free\_vars(t: term): term list}


\manentry{remove\_illegal\_tags(t: term): term}
  Remove all the tags from t that are
illegal according to the tagging restrictions of the direct computation
rules (see page 172 of the book).  The result depends on whether the new
versions of the direct computation rules are in force (see above).

\manentry{simplify(t:term): term}  Does the same thing as the simplification
function used by arith.

\manentry{substitute(t: term; subst: term\#term list): term}
  If the first
component of each pair in subst is a variable, then perform on t the
indicated substitution.

\manentry{replace(t: term; subst: term\#term list): term}
  Like substitute, except
that capture is not avoided.n

\manentry{first\_order\_match(pattern: term; instance: term; vars: tok list):
tok\#term list}
  If the application succeeds, then the result is the most
general first-order substitution s, for those variables in vars that are
free in pattern, such that s applied to pattern is instance.

\manentry{compare\_terms(t1: term, t1: term): bool}
 True if term t1 is
lexicographically less than term t2. False otherwise.  This is a total
ordering of terms and is useful in normalizing expressions.

\manentry{do\_computations(t: term): term}
  From the bottom up, replaced each
tagged term in t by the result of doing the computations indicated by
the tag (see page 172 of the book).

\manentry{compute(t: term): term}
  Head reduce t as far as possible.

\manentry{no\_extraction\_compute(t: term): term\#int}
  Like compute, except that
\tid{term\_of} terms are treated as constants, and the number of reductions
done is also returned.

\manentry{open\_eval(t: term; b: bool; thms: tok list): term}
  Perform the evaluation
defined in the description of the eval rules (see above).

\manentry{eval\_term(t: term): term}
  Apply Nuprl's evaluator to t.  The result is
the same as would be obtained from entering eval mode (using the eval
command) and entering t.

\manentry{instantiate\_def(def\_name: tok; args: term list): term}
  Use args as the
actual arguments to an instance of the definition \tid{def\_name}.  This
function operates by unparsing the arguments textually plugging them
into a definition instance, and reparsing.  This is quite inefficient.

\manentry{instantiate\_trivial\_def(def\_name: tok; arg: term): term}
  The result is
equal to arg, and is an instance of the ``trivial'' definition \tid{def\_name}.
A trivial definition is one which has a single formal parameter that
forms the entire right-hand-side of the template.  This function does
not involving parsing.

\manentry{add\_display\_form:(adder: term -> (tok \# (term list)); t: term):
term}
The operation of this function can be defined recursively.  Suppose the
immediate subterms of t have been replaced by their results under
\tid{(add\_display\_form adder)}, giving t'.  If adder fails on t', then return
t'.  Otherwise, (adder t') is the pair (name,args).  Attempt to
instantiate the definition named by name using args as the actual
parameters.  If this succeeds with a result t''=t', then return t'',
otherwise return t'.  The whole procedure involves only one parsing
operation.

\manentry{remove\_display\_forms(t: term): term}
  Remove all the display forms
(i.e., definition instances) from t.  The result is equal as a term to t
but will be displayed by Nuprl as a term of the base type theory.


\subsection*{For Proofs}

\manentry{frontier: tactic}  For p a proof, fst (frontier p) is the proof list
consisting of the unproved leaves of p.  

\manentry{subproof\_of\_tactic\_rule(r: rule): proof}
  The subproof built by the
refinement tactic that is the body of the rule r.

\manentry{hidden(p: proof): int list}
  The numbers of the hidden hypotheses in the
sequent at the root of p.

\manentry{make\_sequent(dl:  declaration list, hidden: int list, concl:
term): proof}
  The first
argument is the hypothesis list, the second is the list of hypotheses
which are to be hidden, and the last is the conclusion.  The returned
proof has incomplete status, of course.  The function fails if the the
sequent could not ever appear in a Nuprl proof (e.g., if not all
variables are declared).

\manentry{equal\_sequents (p1: proof, p2: proof): bool}
  Sequents are equal when
the variables of corresponding declarations are identical, the terms of
corresponding declarations are alpha-convertible, the same hypotheses
are hidden, and the conclusions are alpha-convertible.

\manentry{main\_goal\_of\_theorem(thm\_name: tok): term}


\manentry{proof\_of\_theorem(thm\_name: tok): proof}


\manentry{extracted\_term\_of\_theorem(thm\_name: tok): proof}



\section{ML Primitives Related to the Library and State}

\manentry{make\_eval\_binding(id: tok; t: term): term} 
The result is
the same (side effects on the eval environment included) as would be
obtained by entering "let id = t;;" in eval mode.

\manentry{execute\_command\_line(line:tok): void}
  \tid{line} is a line of text that one
could type to the command window to the {\tt P>} prompt.  The commands that
can be used are:  load, dump, create, delete, jump, move, scroll, down,
check, view, save, restore, kill, copystate and exit.

\manentry{execute\_command(cmd: tok, args: tok list): void}
  \tid{cmd} is the
name of the command, and \tid{args} is its list of its arguments.

\manentry{create\_def(name:tok, pos:tok, lhs:tok, rhs:term): void}
  \tid{name} is
the name of the def, \tid{pos} is the library position (exactly as one
would type it to the command window), \tid{lhs} is the left-hand side of
the definition (exactly as one would type it into an edit window), and
\tid{rhs} is a term which is unparsed to yield the right-hand side of
the definition.  The free variables of the term are turned into formal
parameters (via the addition of angle brackets).  The function fails if
the set of lhs formal parameters is not equal to the set of rhs formal
parameters.  The generated rhs has outer parens and parens around all
formals added.

\manentry{kill\_extraction(name:tok): void}
  Remove the extraction field from the
theorem whose name is \tid{name}.

\manentry{arity\_of\_def(def\_name: tok): int}
  The number of formal parameters in
the definition \tid{def\_name}.

\manentry{formal\_list\_of\_def(def\_name: tok): tok list}
  A list of the formal
parameters (without angle-brackets) of the definition \tid{def\_name}.

\manentry{rhs\_formal\_occurrences\_of\_def(def\_name: tok): tok list}
  The list of
formal parameters (without angle-brackets) that would be obtained if all
text except formal parameters were removed from the right-hand-side of
the definition \tid{def\_name}.

\manentry{create\_theorem(name: tok; lib\_position: tok; p: proof): void}
  A new
theorem is created in the library with name name, library position given
by \tid{lib\_position} (which should be in the same form as would be given to
the create command), and whose main goal and proof are given by p.

\manentry{create\_ml\_object(name: tok; lib\_position: tok; body: tok): void}


\manentry{kill\_extraction(thm\_name: tok): void}


\manentry{lib(prefix: tok): tok list}
  A list of all names that have prefix as a
prefix and that name an object in the current library.

\manentry{library(():void): tok list}
  A list of the names of all objects in the
current library, in the same order.

\manentry{object\_kind(object\_name: tok): tok}
  One of ``DEF'', ``ML'', ``EVAL'', or ``THM''.

\manentry{rename\_object(old\_name: tok; new\_name: tok): void}


\manentry{object\_status(object\_name: tok): tok}
  One of ``COMPLETE'', ``PARTIAL'', ``RAW'',
or ``BAD''.

\manentry{is\_lib\_member(object\_name: tok): bool}


\manentry{restore\_library\_from\_state(state\_name: tok): void}
  Replace the current
library by a copy of the library contained in the named state.  No other
components of the current state are modified.  (For a description of
states, see the paragraph on the save and restore commands above.)

\manentry{transform\_theorem(x:tok, p:proof): void}
  The first argument is the name
of a theorem, and the second is a proof which is to replace the current
proof of the named theorem.  The new proof cannot have hypotheses, and
its conclusion need not be the same as the current theorem's main goal.
This function is useful in conjunction with tactics and the function
\tid{proof\_of\_theorem}.

\manentry{apply\_without\_display\_maintenance(f:*->**, a:*): **}
  Apply a
function to an argument while disabling the display-maintenance
procedures.  Whenever a substitution into a term is done during the
application, little effort will be made to compute a reasonable display
form for the term.  This means that the term may have all occurrences of
Nuprl definitions removed, leaving a term of the base type theory.
Using this function can result in a ten-fold speedup for computations
(such as symbolic computation) that do alot of substitution.

\manentry{set\_display\_maintenance\_mode(on: bool): void}
  Display-maintenance
procedures become disabled if on is false, and enabled if on is true.

\manentry{display\_message (message: tok): void}
 Displays message in the
command/status window.

\manentry{set\_auto\_tactic(tactic\_name: tok): void}
Set the auto-tactic to be the ML expression represented by \tid{tactic\_name}.

\manentry{show\_auto\_tactic(): tok}
The current setting of the auto-tactic.





\section{Compatibility with Old Libraries}

In order to allow old libraries to be loaded and viewed, it is possible
to restore old versions of some of the rules that have been changed.
The following ML functions can be used to toggle the rule versions used
and to find out which version is in effect.  These functions should not
be considered to be an "official" part of the system; they are provided
only so that old libraries can be viewed.

\manentry{make\_all\_tags\_legal(yes:bool): bool}
  Make the direct computation rules
ignore the tagging restrictions iff yes is true.  

\manentry{all\_tags\_legal(foo:void): bool}
  True iff the tags are ignored.

\manentry{use\_old\_set\_rules(yes:bool): bool}
  Use the old set rules iff yes is
true.

\manentry{old\_set\_rules\_in\_use(foo:void): bool}


\manentry{reset\_rules(foo:void): void}
  Reset the rules to their initial state
(new set rules, all tags legal).






\section{Loading and Compiling ML Code}


Although Nuprl provides library objects for ML code, most users write
their ML programs using a text editor of their choice and store them in
regular files.  There are ML functions (which can be invoked in the
command window in ML mode) that load and compile files of ML code.  The
name of file containing ML code should end in the characters ``.ml''.

There is not a tremendous difference in the performance of compiled versus
uncompiled ML (although it is significant); the main advantage of compiling
is that the ML files can subsequently loaded much more quickly (100
times?).  Also, compiling an ML file is not too much slower than loading an
uncompiled file; in fact, in some Lisps compiling is faster.  

If you load a compiled ML file and get an error having to do with a symbol
being unbound, it means that the compiled file was compiled in an
environment which no longer exists.  This usually happens because a change
is made to an ML file ({\tt load} always gets the newest version of an ML file,
whether or not it is a binary) but a compiled file that depends on it is
left unchanged.

\smallskip

\noindent{\tt load(file\_name: tok; types?: bool): void}.  Look for binary or ML
versions of the file named \tid{file\_name}.  The ML version must exist.  If
the ML version is the newest of the versions, load it; otherwise load the
binary version.  The names for the versions are obtained by adding to
file\_name the appropriate extension (e.g.  ``.ml'' for ML files).  If
\tid{types?} is true, then after loading, a list of the names of all loaded
functions, together with their types, is printed out.  Note: this printout
can blow the control stack in some lisps (e.g. Lucid), possibly corrupting
the state, if the printout is large.

\smallskip

\noindent{\tt compile(file\_name:tok; types?: bool): void}.
Compile the ML version of
\tid{file\_name}.  The compiled version is also loaded into the
ML state.






