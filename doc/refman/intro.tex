\chapter{Introduction}

This document is a reference manual for the version of \Nuprl{}
that runs on Unix/X-window platforms.  Nuprl will also run on 
Symbolics Lisp machines and on Sun workstations running the SunView
window system, although these versions of Nuprl are now considered
obsolete; see the files {\tt doc/symbolics} or {\tt doc/sunview} for
further information on these versions.

In addition to the system proper, this manual also documents the
tactic collection which is loaded into Nuprl by default.  This
documentation is somewhat sketchy.  Also, many of the mathematics
libraries built at Cornell using Nuprl have used tactic collections
different from the one described here.  The chapter on the tactic
collection also gives some examples of their use.

It is assumed that the reader is familiar with Nuprl at the level of
the ``tutorial'' chapters of the book {\em Implementing Mathematics
with the Nuprl Proof Development System}, Constable {\em et al} ---
henceforth referred to as ``the book''.  The tutorial part of the book
introduces the system and its type theory and provides a number of
examples.  The ``reference'' portion, excluding the parts of Chapter 8
on the type system and its semantics, is superseded by this reference
manual.  Chapter 9 in the reference portion also contains some useful
examples and discussions of tactic writing that are not reproduced
here.  The ``advanced'' portion of the book deals with application
methodology, gives some extended examples of mathematics formalized
in Nuprl, and also describes some extensions to the type theory which
have not been implemented.

This manual assumes that Nuprl will be running on a machine that has a
three-button mouse.  However, except for a few inessential
user-interface features, anything that can be done with the mouse can
be done by other (although possibly less convenient) means.

\section{System Overview}

\label{Overview}

One interacts with Nuprl using a collection of windows.  These windows are
at the ``top-level'' in the X environment, so they can be managed
(positioned, sized, etc.) in the same way as other top-level applications
such as X-terminals.  Consult your local documentation for details on
general window management.  Creation and destruction of Nuprl windows, and
manipulation of window contents, is done solely via commands interpreted by
Nuprl.  Nuprl will receive mouse clicks and keyboard strokes whenever the
the input focus is on any of its windows.  Exactly one window is
``active'' at any given time; this window is identified by the presence of
Nuprl's cursor which appears as a small solid black rectangle.  The
specific location of the cursor determines the semantics of keyboard
strokes and mouse clicks, and is independent of the location of the mouse.

When Nuprl is first invoked it will automatically create two windows, one
called the {\em command window} and the other the {\em library} window.  These
two windows remain throughout the session and new ones cannot be created.
The command window, when active, accepts lines of text terminated by a
carriage return.  The effect of entering a line of text depends on the
{\em mode} of the command window.  The mode is indicated by the prompt
printed in the window at the beginning of the current line.  There are five
modes (details are given below): {\em command mode} (prompt {\tt P>}) for
entering basic commands, {\em ML mode} for entering ML expressions to be
evaluated, {\em eval mode} for entering expressions to be evaluated as type
theory programs or as bindings of programs to identifiers, {\em Lisp mode} for
entering Lisp expressions to be evaluated, and {\em instantiation mode} for
entering names of definitions to be instantiated.

The library window displays the current contents of Nuprl's library.  The
library is a list of named objects.  An object is either a definition, a
theorem together with its proof, or a text object containing either ML code
or expressions appropriate for eval mode.

There are two other kinds of windows.  Both are for editing objects in the
library.  Such windows can be created using commands entered in the command
window.  One kind of editing window is for text editing.  These windows are
used for editing definitions, goals of theorems, and rules
applied in proofs.  The other kind of window is a {\em refinement-editor}
window, which is used to view and manipulate proofs as tree-structured
objects.

Proofs are trees whose nodes consist of sequents and justifications.  A
justification is either the name of a rule, {\em empty}, or {\em bad}
(ill-formed or inapplicable).  Rules are either {\em primitive} rules,
which are rules directly related to the type theory, or the {\em tactic}
rule, which consists of the text of an ML program called a {\em tactic}.
Proofs are built and modified by textually editing justifications and by
applying {\em transformation tactics}.

\section{Summary of Changes}

With the exception of the addition of a tactic collection, the changes
that have been made to Nuprl since the book was written have been
relatively minor.  The main change was the addition of an interface to
X-windows.  There were also changes and additions to the rules,
addition of several user interface features, and implementation of a
substantial set of new ML ``utilities''.  The changes to the set of
rules are as follows.
\begin{itemize}
\item Modification of the  {\em direct-computation}, {\em
extensionality}, {\em set} and {\em arith} rules.
\item Addition of ``reduce'' rules for decision terms (\tid{atom\_eq}
etc). 
\item Addition of the {\em monotonicity}, {\em division}, {\em
thinning}, {\em eval} and {\em reverse-direct-computation} rules.
\item ``Experimental'' addition of ``simple'' recursive types and 
the ``object'' type (see Appendix~\ref{experimental-types}).
\end{itemize}
Some of the changes to the user interface are:
\begin{itemize}
\item additional bindings of keys to special functions,
\item ability to use ``special'' characters such as $\forall$ in
definitions,
\item new features involving the mouse, and
\item some new commands.
\end{itemize}
The most significant new ML utilities deal with printing out proofs
and libraries.  These can be printed out in a reasonably readable
textual format, or as a source file for Latex.


\section{Learning to Use Nuprl}

One way to get started learning how to use the Nuprl system is to try
some of the examples in Chapters 3 and 4 of the book.  These examples
give a tutorial-like introduction to basic commands for editing text
and proofs.  Before doing this, though, Section~\ref{Keys} below on
special keys and mouse-clicks should be read.

Many of the examples in Chapters 3 and 4 have been worked out and
stored in the library {\tt lib/basics/lib}.  In particular, the
library contains basic definitions for logical notions and proofs of a
few simple logical facts and of the existence of integer square roots.
The library makes use of one of the special fonts distributed with
Nuprl.  If these fonts have not been installed at your site, you can
use the library ``ulib'' in place of ``lib'' above (the ``u'' is for
``ugly'', since the library replaces, for example, the symbol for the
universal quantifier by the string ``All'').

To load this library, one can type
\begin{center}
 \tt load fully top from lib/basics/lib
\end{center}
(followed by a carriage return) in the command window.  In English, 
this command means: {\em load the library objects from the file
lib/basics/lib into the current library starting at the top, fully
reconstructing all proofs}.


\section{Interactions with Lisp}

Nuprl is a Lisp program, and a few Nuprl-related operations, in
particular starting Nuprl, must be done in Lisp.  This section
describes all such operations.

The following assumes that Nuprl has been correctly installed and also that
a Lisp image containing Nuprl has been created.  If a special Lisp image
has not been created, then you will need to follow the instructions on
loading Nuprl into Lisp in the document \tid{doc/installing-nuprl}.  Once
the Lisp image containing Nuprl is running, subsequent input will be
interpreted as Lisp expressions to be evaluated.  To start Nuprl, evaluate
\begin{quote}
  \tt (nuprl "{\em host}")
\end{quote}
where {\em host} is the name of the machine where  Nuprl's
windows are to appear.  
The evaluation of this function terminates when the
Nuprl session is exited.   

If there is a file named {\tt nuprl-init.lisp}\footnote{The filename
extension for lisp source files, {\tt lisp} in this case, depends on the
Common Lisp and kind of machine being used.} in your home directory and if
Nuprl has not been started since it was loaded or since the last time it
was reset (see below), then starting it will cause the file to be
immediately loaded.  This allows various window-system customizations to be
done.  The file \tid{sys/sample-nuprl-init-for-X}'' gives an example of some
useful customizations for X-windows.  Details on possible customizations
are given below.

Nuprl runs in the ``nuprl'' package.  All symbols entered in Lisp will
be interpreted relative to this package.  The package contains all the
symbols of Common Lisp, but does {\em not} contain the various
implementation-specific utilities found in the package ``user'' (or
``common-lisp-user'').  To refer to these other symbols, either change
packages using {\tt (in-package "USER")}, or explicitly qualify the
symbols with a package prefix.  For example, to use the Lucid (Sun)
Common Lisp function \tid{quit} to quit lisp, evaluate {\tt
(user::quit)}.  (Note the double-colon.)  If you change packages, you
can change back to the Nuprl package using {\tt (in-package "NUPRL")}.


Most Lisps allow computations to be interrupted.  This is usually done by
entering \CTRL{C} (that is, control-C) in the Lisp window.  This will cause
Lisp to enter its debugger, from which the computation can be resumed or
aborted (in Lucid, \tid{:a} aborts and \tid{:c} resumes).  Aborting Nuprl
is almost always safe.  When Nuprl is restarted, the state should be
exactly as it was when Nuprl was killed, except that any computations
within Nuprl will have been aborted.  To restart Nuprl it is sufficient to 
evaluate \tid{(nuprl)}.

It is possible (though not too likely) for Nuprl to get an error which
will cause the Lisp debugger to be entered.  This will likely be due
to a user improperly setting an X-window parameter using the
customization facility described below.  If this happens, the mistake
must be fixed and Nuprl reinvoked.

On rare occasions Nuprl's window system may get an error from which it is
impossible to recover by just aborting and reinvoking.  In this case, abort
out of Nuprl, and enter the Lisp form {\tt (reset)}.  This should fix the
problem.  It will cause Nuprl's window system to be reinitialized, but
otherwise the state will be left intact.

For those unfamiliar with Lisp-based systems, if the system appears to be
inexplicably stuck, check the window running Lisp to see if Lisp is
garbage-collecting.



\section{Customizing Nuprl}

The Lisp function \tid{change-options} can be used to set various
parameters affecting Nuprl's window system.  An appropriate place to call
this function is from your init file.  The default values are reasonable,
so you need not have an init file.

The \tid{change-options} function takes an argument list consisting of
keywords and associated values.  For example, to set the options :host and
:font-name to ``moose'' and ``fg-13'' respectively, put the form
\begin{quote}
\tt  (change-options :host "moose" :font-name "fg-13")
\end{quote}
in your init file.  The options, together with their default
values (in parentheses) are given below.
\begin{quote}
{\tt{}:host} ({\tt{}NIL}).  The host where Nuprl windows should appear.  

{\tt{}:title-bars?} ({\tt{}NIL}).  If {\tt{}T}
then Nuprl will draw its own title bars.

{\tt{}:host-name-in-title-bars?} ({\tt{}T}).  
If {\tt{}T}, then the title of each window will
include a substring indicating what host the Lisp process is running on.

{\tt{}:no-warp?} ({\tt{}T}).  If {\tt{}T} then Nuprl will never
warp the mouse.  (Mouse
warps apparently annoy some users.)  In environments where the position of
the mouse determines input focus, setting this option to \tid{NIL} will
guarantee that Nuprl retains input focus when windows are closed.

{\tt{}:frame-left} ({\tt{}30}), {\tt{}:frame-right} ({\tt{}98}), 
{\tt{}:frame-top} ({\tt{}30}), {\tt{}:frame-bottom}
({\tt{}98}).  Each of these should be a number between 0 and 100.  They give
the boundaries, in terms of percentage of screen width or height, of an
imaginary frame within which Nuprl will attempt to place most new windows.

{\tt{}:font-name} ({\tt{}"fixed"}).  The name (a string) of a font to use
for the characters in Nuprl windows.  The font must be fixed-width.  The
default is reasonable and should always be available, but it does not
contain the special characters used in many of the libraries built at
Cornell.  These libraries will have appearance problems with wimpy fonts,
although they should be correct.  Some of the better fonts are as follows.
fg-25, fg-30, fgb-25, fg-40: these have all the special characters except
centre-dot.  fg-13, fg-20, fgb-13, fr-25: all except integral, gamma,
delta, plus-minus, and circle-plus.  These fonts are not supplied in recent
releases of the X-window system, but they are included in the Nuprl
distribution.  To use these fonts they must have been installed correctly
and the X-server must be made aware of them --- see the file
\tid{doc/installation} for details on how to do this.

{\tt{}:cursor-font-name} ({\tt{}"cursor"}), {\tt{}:cursor-font-index}
({\tt{}22}).  The name of the font to use for the mouse cursor when it is
over a Nuprl window, and an index into that font.  The default font should
always be available.

{\tt{}:background-color} ({\tt{}"white"}).  The color for the
background in Nuprl windows.  The value must be a string argument
naming a color.  Any color in the X-server's default colormap may be
used.  Nuprl will get a Lisp error (entering the Lisp debugger) if the
color does not exist.

{\tt{}:foreground-color} ({\tt{}"black"}).  The color for characters 
etc.\ in Nuprl windows.

  
\end{quote}


\section{Mouse Buttons and Special Keys}

\label{Keys}

Some of the keys on the keyboard are bound to special functions.  Many of
these keys will be referred to symbolically.  This section describes the
correspondence between the symbolic names and actual keys.  In order to
accommodate a variety of preferences, many of the names correspond to
several keys.

Following are some of the key bindings, together with a brief description
(see below for details).
\begin{quote}\protect

\COPY{} = \CTRL{C}.  Copy text from buffer.

\INS{} = \CTRL{I}.  Instantiate a definition.

\KILL{} = \CTRL{K}.  Kill a character, definition or selected region.

\EXIT{} = \CTRL{D}.  Exit a window or ML or Eval mode.

\DEL{} = Rubout.     Delete the preceding character.

\ERASE{} = \CTRL{U}.  Erase the current input line (command window only).

\MODE{} = \CTRL{M}.  Toggle ``bracket mode''.

\TRANSFORM{} = \CTRL{T}.  Run a transformation tactic.

\PRINT{} = \CTRL{O}.  Output the contents of
the active window to a file.  

\CMD{} = Tab.    Make the command window active.

\CR{} = Return.  Carriage return.
\end{quote}

The rest of the special keys are organized as a keypad with the 
following layout
\begin{center}
\begin{tabular}{|c|c|c|} \hline
\DIAG & \UP & \JUMP \\ \hline
\LEFT & \LONG & \RIGHT \\ \hline
\SEL & \DOWN & \\ \hline
\end{tabular}
\end{center}
These keys are used to move Nuprl's cursor around and to select text.
Almost everything that can be done with these keys can also be done using
the mouse.  On  machines with ``meta'' keys,
symbolic keys are mapped to two simulated keypads: when the ``meta'' key is
held down, the leftmost block of 9 keys (q-w-e-a-s-d-z-x-c) and the
rightmost block (u-i-o-j-k-l-m-,-.)  are both interpreted as keypads.

As an alternative to the keypad there is a set of Emacs-like bindings that
do not involve the ``meta'' key.  These are given below, along with a
short description of their main function.  
\begin{quote}\protect
\SEL\ = \CTRL{S}.  Select (a template, an endpoint of a piece of text, \ldots).

\DOWN\ = \CTRL{N}.  Move the cursor down (``next'').

\LEFT\ = \CTRL{B}.  Move the cursor left (``back'').

\LONG\ = \CTRL{L}.  Amplify the next cursor movement.

\RIGHT\ = \CTRL{F}.  Move the cursor right (``forward'').

\DIAG\ = \CTRL{Q}.  Move ``diagonally'' up a proof tree.

\UP\ = \CTRL{P}.  Move the cursor up (``previous'').

\JUMP\ = \CTRL{J}.  ``Jump'' to another window.
\end{quote}

By holding down both the ``control'' and ``meta'' keys one can obtain some
special printing characters (such as some of the Greek letters), if they
are available in the current font.  These characters should only be used on
the left hand sides of the templates in Nuprl definitions.  To see a list
of these key bindings, right-click the mouse, and a list of them will
appear in the command window.  If your machine does not have a meta key
then you cannot use special characters.

Additional key bindings can be defined in your init file by using the
function define-key-binding.  This requires a bit of knowledge about how
Nuprl deals with characters internally.  See the file
doc/implementation-notes for more details on this.

Slight variants of the functions of the keys \SEL\ and \JUMP\ are bound to
the left and middle mouse buttons, respectively.  Henceforth, these buttons
will be referred to as mouse-\SEL\ and mouse-\JUMP.  See below for more on
the use of these buttons.


