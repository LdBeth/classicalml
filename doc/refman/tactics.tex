% -*- Default-character-style: (:FIX :ROMAN :VERY-LARGE); Mode: Tex -*-


\chapter{Basic Tactics}

\label{basic}


A collection of ML files is automatically loaded into Nuprl whenever the
system is loaded.  Some of these files contain the implementation of the ML
functions described in the book.  The remaining files contain a fairly
large collection of tactics and related functions.  This tactic collection
should not be regarded as part of Nuprl.  Many tactic collections have been
developed at Cornell over the last several years; this one was chosen
because it is the most recent large collection that has been reasonably
documented.  It is not, however, free of problems.  Some of the tactics
have minor bugs, and others could be easily made much more effective.

Serious users of Nuprl will want to build on and modify the basic tactic
collection.  For those who wish to do so, the relevant files are named in the
file \tid{nuprl/ml/load}.  Users who want a detailed knowledge of these tactics
will also have to consult these files, since this document does not contain a
full desription of the tactic collection.

Most of the tactics (and related objects) in the Nuprl collection will be
at least mentioned in this document; the ones that are omitted are mostly
simple variants of objects that are mentioned, or have turned out not to be
very useful in practice.  An example of the use of the tactics can be found
in the library \tid{nuprl/lib/saddleback/new-lib}.  Expanding this library
does not require that any extra ML files be loaded.

The tactic collection described in this document is almost identical to the
Nuprl 2.0 collection.  Many new tactics have been written since Nuprl 2.0
was released, but we have not had the time to package and document these
tactics for use elsewhere.  We hope to be able to do this in the near
future, though.  The remainder of this document is the same as the one
describing the 2.0 tactics except for a brief section at the end which
describes a few additions and fixes.

Most of this document has the form of a user's manual, although many of the
descriptions will omit details that are uninteresting but necessary for
actual use of the described function.  A typical description will be headed
by a schematic application that may contain type information.  For example, a
description of the function {\tt destruct\_hyp} might be introduced by
\begin{quote}
\tt       destruct\_hyp (i:int) (p:proof): tok\#term.  
\end{quote}
This indicates that the function has type \tid{int -> proof -> tok\#term}.  The
parameters (\tid{i} and \tid{p} in the example) will often be referred to
in the description.

Many of the tactics described perform some kind of rewriting on a component
of a sequent.  There are usually several ways to apply a particular kind of
rewriting; for example, one can apply it just to the conclusion, or to some
hypotheses, or to all hypotheses and the conclusion.  Usually, only one of
these will be mentioned (usually the one which applies rewriting everywhere).
For example, only \tid{Eval} will be described, although there are also tactics
\tid{EvalHyp} and \tid{EvalConcl}.


\section{Definitions}

\label{definitions}

To take full advantage of the tactic collection,
the user must follow a simple convention in making
definitions.  As pointed out in the previous
chapter, although Nuprl's definition facility
allows considerable flexibility in making notations for
terms, it cannot reasonably serve as an abstraction
mechanism.  A solution to this deficiency is to
define objects via extraction.  Using this scheme,
a complete definition consists of two library
objects: a theorem object whose extraction is the
defined term, and a definition object which gives a
display form for the {\tt term\_of} term which
denotes the extraction.  Sometimes the defining
theorem also serves to give a type to the defined
term; other times, a separate theorem is used for
this.  The library objects associated with a
definition are related by a naming convention
(involving trailing underscores).

There are two styles of definition in this scheme.  We will describe
these using two simple examples.  The first style of definition is
exemplified by the definition of a function which takes the maximum of
two natural numbers.  The definition object for this function is
called {\tt N\_max}, and its body is
\begin{Numath}
\Nu{max(<m>,<n>) == term\_of(N\_max\_)(<m>)(<n>).}
\end{Numath}%
The theorem {\tt N\_max\_} has statement
\begin{Numath}
\Nu{>> N -> N -> N,}
\end{Numath}%
and is proved by explicitly introducing the desired object, using the tactic
invocation 
\begin{Numath}
\Nu{(ExplicitI '\mlambda{}m n. less(m;n;n;m)' ...).}
\end{Numath}%

The second style of definition is exemplified by the definition of the length
function (for lists).  The definition object {\tt length} is
\begin{Numath}
\Nu{|<l>| == term\_of(length\_)(<l>).}
\end{Numath}%
The theorem {\tt length\_} has statement
\begin{Numath}
\Nu{>> Object}
\end{Numath}%
and extraction
\begin{Numath}
\Nu{\mlambda{}l. list\_ind(l; 0; h,t,v.v+1).}
\end{Numath}%
The type of the theorem extracted from is the trivial type {\tt Object}
because it is convenient to have the polymorphism of {\tt length} implicit.
(The type {\tt Object} never appears except in these kinds of definitions.
See Appendix~\ref{experimental-types} for more information on {\tt Object}).  In other words,
instead of taking as arguments a type and a list over that type, the length
function as defined above just takes a list as an argument.  The typing lemma
for this definition is called {\tt length\_\_} and is
\begin{Numath}
\Nu{\mforall{}A:U1. \mforall{}l:A list. |l| in N}
\end{Numath}%

In summary, an instance of a definition is actually an application of a 
{\tt term\_of} term to some arguments.
Definition objects in the library
should be
regarded as incidental additions that make displayed terms easier to read.

\section{Terms}

There is a large set of functions for dealing with terms.  
Most of these functions are for
analyzing, constructing, or recognizing terms.  Only three of them
will be described
here.


{\tt get\_type ($p$:proof) ($t$:term).} The result
is a type $T$ for $t$, such that the autotactic can
(almost always, in practice) prove that $t$ is a
member of $T$ under the hypotheses of the sequent
$p$.  The algorithm that computes $T$ uses a
function {\tt g} that computes a type given a term
$t$ and a typing environment $e$ (an association
between Nuprl variables and types).  This function
uses a simple recursive algorithm.  If $t$ is a
variable, the result is obtained by looking up $t$
in $e$.  If $t$ is a canonical term, the result is
computed from types for the immediate subterms.
This is not alway possible (in which case an
exception is raised), as when $t$ is a
lambda-abstraction.  See the discussion of the
membership tactic below for an explanation of why
this kind of failure tends not to be a problem in
practice.  When $t$ is a non-canonical term, the
first step is to compute a type $T$ for its
principle argument.  Head normalization and
stripping off of set types is then repeatedly
applied to $T$ until a type $T'$ is obtained that
is canonical and not a set type.  If $t$ is an
application $f(a)$, then $T'$ must be a function
type {\tt $x$:$A$->$B$} (else failure), and the
result is $B[a/x]$.  If $t$ is of the form {\tt
decide($a$;$u$.$b$,$v$.$c$)}, then $T'$ must be a
disjoint union {\tt $A$|$B$}, and the result is {\tt g}
applied to $b$ and the environment $e$ updated with
an association between $u$ and $A$; or if this
fails, an analogous application to $c$.  The {\tt
spread} case is analogous.  If $t$ is an induction
form, then an attempt is made to compute the result
just from the base case.  Finally, if $t$ is of the
form {\tt term\_of($a$)}, then the result is the
main goal of the theorem named $a$.

A special case is when $t$ is an instance of a definition of the second kind
described in Section~\ref{definitions}.  In such a case, there is a theorem that
is a universally quantified statement ending with a term of the form {\tt
$a$~in~$A$}.  Matching of $a$ against $t$ is used to compute terms for
instantiating the quantified variables appearing in $A$, and the resulting
instance of $A$ is returned as the type of $t$.

{\tt match\_part\_in\_context $f$ $A$ $t$ $p$ $l$}.  Find a subformula
$A'$ of $A$, and a list of terms that yield $t$ when substituted in $A'$ for
the variables that are bound by the universal quantifiers in $A$ whose scope
contains $A'$.  The result is the list of terms.  The subformula $A'$ is one
which can be obtained by descending through conjunctions, through universal
quantifiers, through implications via the consequent, and finally by
applying the function $f$, which should be a term destructor.
First-order matching is used to compute substitutions for the free variable
occurrences of $A'$ that are
bound in $A$.   
If matching does not provide instantiating terms
for all the universal quantifiers encountered on the descent to $A'$,
then type information is used.  In particular, for each universally quantified
variable $x$ which has been instantiated with a term $a$, a type is computed for
$a$ using {\tt get\_type $p$ $a$}, and this type (or certain derived types, if
necessary) is matched against the type obtained from the appropriate universal
quantifier.  After all possible type information has been used, the term list
$l$ is used for any further required instantiating terms.  

The use of type information is important for definitions that use implicit
polymorphism.  As a simple example, consider the function {\tt tl} that
computes the tail of a list.  Suppose that it has the typing lemma
\begin{Numath}
\Nu{\mforall{}A:U1. \mforall{}l:A list. tl(l) in  A list.}
\end{Numath}%
In computing the type of a particular application {\tt tl($s$)},
the matching of {\tt tl($s$)} against {\tt tl(l)} gives an
instantiation only for {\tt l}.  If a type for $s$ can be computed, and this
type is (or can be reduced to) a list type, then matching against the declared
type {\tt A~list} for {\tt l} gives an instantiation for {\tt A}.  Instantiating
the lemma with the computed terms yields a type for {\tt tl($s$)}.

{\tt tag\_redices ($t$:term)}.  This tags redex-contractum pairs in $t$ so that
direct computation will result in all the contractions being performed.  This
function can be updated from ML objects in the library (by adding to a global
list).  This allows the user to make some of the tactics that use reduction take
into account definitional extensions to the redex-contractum relation.


\section{Computation}

\label{computation}

There are many tactics which are based on the evaluation and direct computation
rules.  These include the tactics for 
definition expansion.

{\tt Eval}.  Perform evaluation on the conclusion and all hypotheses, 
expanding {\tt term\_of} terms whenever required for computation to proceed.

{\tt EvalOnly {\id{names}}}.  Like {\tt Eval}, except that terms of the form
{\tt
term\_\-of($x$)} are only expanded if $x$ is a member of the list \id{names}.

{\tt EvalExcept {\id{names}}}.  Like {\tt Eval}, except that terms of the
form {\tt term\_\-of($x$)} are treated as constants when $x$ is a member of the
list \id{names}.

{\tt EvalSubtermOfConcl $b$ \id{names} $P$}.  Evaluate a subterm of the
conclusion that satisfies the predicate $P$.  The arguments $b$ and \id{names}
have the same purpose as in the evaluation rule.

{\tt CC}.  Compute the conclusion to head normal form.

{\tt ComputeEquands}.  If the conclusion is of the form {\tt $a$=$b$~in~$A$}
then head-normalize the ``equands'' $a$ and $b$, else fail.

{\tt Reduce}.  Repeatedly perform reductions of redex-contractum
pairs wherever the legal-tagging restrictions permit it.

{\tt HypModComp($i$:int)}.  It can require excessive
computation time to demonstrate that a hypothesis
is equal (as a type) to the conclusion of a sequent
by completely normalizing both terms and then
checking that the results are identical.  {\tt
HypModComp} provides an approximation to complete
normalization that is usually much faster.  It
proceeds by applying reductions to a subterm in the
hypothesis and a corresponding subterm in the
conclusion.  Initially, the subterms are the entire
respective terms.  At each stage, it applies the
smallest number of reductions such that the two
resulting subterms are either instances of the same
definition, or are in head normal form with the
same head.  If this is impossible, the tactic
fails, otherwise the procedure is repeated on the
immediate subterms of the resulting subterms.  An
important application of this tactic is in the
inclusion tactic described below.

There are several tactics that are used for {\em unfolding} 
(expanding) and {\em folding} (the inverse of unfolding) definitions.  Because
definitions are applications of {\tt term\_of} terms to some arguments,
unfolding a definition occurrence in a sequent requires using direct computation
to replace the appropriate {\tt term\_of} term by its denotation and to do 
reduction steps until all the arguments to the definition instance are
substituted into the body of the definition.  

{\tt Unfold \id{names}}.  Unfold all definitions whose name is in the list
\id{names}.

{\tt Fold \id{names}}.  Add as many instances of the named definitions as
possible.  This is not alway easy, since arguments to a definition instance
can disappear when the definition is expanded.  This is often the case for
arguments that are types.   For example, the length function given in
Section~\ref{definitions} could have been defined to be 
\begin{Numath}
\Nu{\mlambda{}A. \mlambda{}l. list\_ind(l; 0; h,t,v.v+1),}
\end{Numath}%
which has type
\begin{Numath}
\Nu{A:U1 -> l:(A list) -> N.}
\end{Numath}%
To fold a term 
\begin{Numath}
\Nu{list\_ind($l$; 0; h,t,v.v+1)}
\end{Numath}%
into an application of the length function, the type of $l$ has to be computed.

{\tt UnrollDefs \id{names}}.  For each \id{name} in \id{names} and each
instance of the definition named \id{name}, do the following.  Unfold the
definition, then fail if the result is not a redex that
is of the form {\tt list\_ind(\ldots)} or {\tt rec\_ind(\ldots)}, otherwise
contract the redex and attempt to fold it to an instance of the definition
\id{name}.  For
example, unrolling {\tt |h.t|} gives {\tt 1+|t|}.

\section{Tacticals}

A {\em tactical} in LCF is a function which is used for combining tactics.  The
basic LCF tacticals have analogues in Nuprl.  

{\tt $T$ THEN $T'$}.  Apply the tactic $T$.  If it fails, then fail, otherwise apply
$T'$ to the subgoals generated by $T$.  This tactical has several variants
which incorporate useful restrictions on the kind of subgoal $T'$ is applied
to.  {\tt $T$~THENS~$T'$}, when applied to a proof $p$, only applies $T'$ to
subgoals that have the same conclusion as $p$.  {\tt THENO} applies its second
argument only to goals that have a {\em different} conclusion.  {\tt THENM} and
{\tt THENW} apply their second argument only to goals that are not membership
goals or well-formedness goals respectively.  (A membership goal is one
whose conclusion is of the form {\tt >>~$t$~in~$T$}, and a well-formedness goal
is one whose conclusion is of the form {\tt >>~$T$~in~U$i$}.)  Because many
tactics have actions that are well-defined except for the generation of
``junk'' subgoals, the variants are used extensively for combining these tactics
(both at the top level in proof construction, and in the construction of other
tactics).

{\tt Repeat $T$}.  Apply $T$ until it fails or ceases to make progress.  This
tactical has variants analogous to the variants of {\tt THEN}.

{\tt $T$ ORELSE $T'$}.  Apply $T$.  If it fails, apply $T'$.


\section{Some Derived Rules}

\label{derived}

Many of the tactics in this section generate unwanted subgoals.  These are
usually membership goals, or can be proven by some trivial propositional
reasoning.  In practice they can almost always be proved by the autotactic, and 
so the tactics will be described as though they did not generate any
``junk''. 

For each primitive inference rule in Nuprl, there is a corresponding tactic.
Such tactics are useful partly because they compute many of the parameters required
by the corresponding rule.  Some of these parameters are easily computed, 
such as those
giving names for new variables, and many are not, such as those which supply
types for certain subterms of the goal.   Many of these tactics also incorporate
computation. For rules which require that a hypothesis or the conclusion be a
canonical type, the corresponding tactic will head-normalize the appropriate
term before attempting to apply the rule. 

{\tt ILeft}.  If the conclusion is or computes to a disjoint union, then apply
the introduction rule which picks the left disjunct.  {\tt IRight} is analogous.

{\tt ITerms $l$}.  If the conclusion is existentially quantified (using product
types or set types), then introduce the terms in the list $l$ as witnesses for
the existentially quantified variables.

{\tt I}.  If the previous three tactics do not apply, and if the conclusion is a
canonical type that is not an equality or {\tt void}, then apply the
introduction rule for the type.

{\tt EqI}.  If the conclusion is of the form {\tt $t$~in~$T$} or {\tt
$t$=$t'$~in~$T$}, where $t$ is not a variable and, in the latter case, $t$ and
$t'$ have the same outermost term constructor, then apply the appropriate
equality-intro rule.  This will often involve computing a type.  For example,
when applied to the goal 
\begin{Numath}
\Nu{>>~$f$($a$) in $T$,}
\end{Numath}%
{\tt EqI} must compute a
functional type for $f$ in order to apply the equality-intro rule for
application.  If the type
computed for $f$ is {\tt $x$:$A$->$B$}, but $B[a/x]$ is not identical to $T$,
then no equality-intro rule applies.  In this case, {\tt EqI} will assert 
\begin{Numath}
\Nu{$f$($a$) in $B[a/x]$.}
\end{Numath}%
To prove that this implies the original goal, {\tt EqI} calls {\tt Inclusion},
which is described in Section~\ref{autotactic}.

{\tt EOn $t$ $i$}.  Instantiate with $t$ the universally quantified formula
in hypothesis $i$.

{\tt E $i$}.  If the type in hypothesis $i$ is not universally quantified,
apply the appropriate elimination rule.

{\tt Unroll $i$}.  If hypothesis $i$ declares a variable $x$ to be of a list or
recursive type, then ``unroll'' the type.  If the type is a recursive type, the
main step is to apply the ``unroll'' rule for recursive types.  If it is a list
type, then there are two main subgoals.  The first is essentially the goal with
$x$ replaced by {\tt nil}, and the second is the goal with $x$ replaced by the
cons {\tt $h$.$t$} for $h$ and $t$ new variables.

{\tt SubstFor $t$}.  If $t$ is of the form {\tt $a$=$b$~in~$A$}, substitute $b$
for $a$ in the conclusion.  There is an analogous tactic, {\tt SubstForInHyp},
which works on hypotheses.

There are also slightly generalized versions of some of the elimination rules.
For example, {\tt RepeatAndE} breaks up a hypothesis which is a conjunction,
making each conjunct into a new hypothesis.  The work done by some of these
generalizations can be somewhat substantial.  As an example, consider
the elimination rule for dependent product types.  If a hypothesis declaring
some variable $x$ to be in a type of the form {\tt $y$:$A$\#$B$}
is eliminated, then the hypothesis list is extended by three new hypotheses: 
\begin{Numath}
\Nu{$y$:$A$, $z$:$B$, $x$=<$y$,$z$> in $y$:$A$\#$B$,}
\end{Numath}%
and in the conclusion {\tt <$y$,$z$>} is substituted for $x$.  The tactic {\tt
GenProductE} generalizes this rule to an arbitrary number of repeated
products.
For an $n$-fold product, $n+1$ new
hypotheses are generated, the last of which is an equation between $x$ and an
$n$-tuple.  Simply chaining together an appropriate sequence of applications of
the elimination rule will generate $n$ equalities, which must be collapsed using
substitution.  The tactic uses another method which is slightly better.  In
either case, new membership subgoals are generated along the way, and so
applications of the tactic can be much slower than one might expect.  

A useful way to apply elimination rules is with {\tt OnVar}.

{\tt OnVar $x$ ($T$:int->tactic)}.  If $x$ is declared in
the hypothesis list, apply $T$ to that hypothesis.  Otherwise, repeatedly
apply introduction rules as long as the conclusion is a universal quantification
or an implication, or until $x$ becomes declared, in which case apply
$T$ to the hypothesis declaring it.

There are several tactics that perform generalization, replacing a term in the
conclusion by a new variable.

{\tt LetBe $x$ $a$ $A$}.  If the goal is {\tt >>~$B$}, then the subgoal
generated is
\begin{Numath}
\Nu{$x$:$A$, $x$=$a$ in $A$ >> $B[x/a]$}
\end{Numath}%
(regarding a subgoal {\tt >>~$a$~in~$A$} as a ``junk'' subgoal).

{\tt TypeSubterm $a$ $A$}. If the goal is {\tt >>~$b$ in $B$}, then the subgoal
generated is
\begin{Numath}
\Nu{$x$:$A$ >> $b[x/a]$ in $B$.}
\end{Numath}%

{\tt ETerm $a$}.  This is similar to {\tt LetBe}, except that the type $A$ for $a$
is computed, and elimination is done on the resulting new declaration.

There are some simple derived rules for equality reasoning.

{\tt DestructEq $f$ $i$}.  If hypothesis $i$ is of the form {\tt
$a$=$b$~in~$A$}, then compute a type $A'$ and add a new hypothesis 
\begin{Numath}
\Nu{$f$($a$) = $f$($b$) in $A'$.}
\end{Numath}%
The term $f$ must be a lambda-abstraction, and should be a destructor function
(such as projection from pairs).

{\tt SplitEq $t$}.  If the conclusion is of the form {\tt $a$=$b$~in~$A$}, then
generate two subgoals, with conclusions {\tt $a$=$t$~in~$A$} and {\tt
$t$=$b$~in~$A$}.

There are two frequently used tactics related to the set type.

{\tt Unhide}.  If the conclusion is a squashed type (of the form
$\Set{Int}{A}$ for some type $A$)
then the single subgoal is identical to the goal,
except that no hypotheses are hidden (a hidden hypothesis is one that cannot be
used until it becomes unhidden---hidden hypotheses are created only by the
set-elim rule). 

{\tt Properties $l$}.  For each term $t$ in the list $l$, compute a type for
$t$, head-normalize the type, and if a type of the form {\tt \{$x$:$A$|$B$\}} is
obtained, then add $B[t/x]$ as a new hypotheses.

We end this section with a few miscellaneous tactics.

{\tt AbstractConcl $t$}.  If the conclusion is $A$, then create a new variable
$x$ and produce a subgoal with conclusion
\begin{Numath}
\Nu{(\mlambda{}$x$. $A[x/t]$) ($t$).}
\end{Numath}%
This is useful in the application of certain lemmas which quantify over
predicates, such as those expressing induction schemes.

{\tt BringHyps $l$}.  ``Bring'' some hypotheses (specified by the numbers in $l$) to
the conclusion side of the sequent, adding them to the old conclusion as
antecedents of an implication or as universal quantifiers.  As an example,
bringing the last hypothesis of
\begin{Numath}
\Nu{$x$:$A$ >> $B$}
\end{Numath}%
results in
\begin{Numath}
\Nu{>> $x$:$A$->$B$.}
\end{Numath}%

{\tt NonNegInd $x$ $i$}.  If the type in hypothesis $i$ is (or computes to) a
subrange of the integers, then use it to calculate an integer lower bound for
the variable $x$ declared by the hypothesis.  If the lower bound is
non-negative, then do induction on $x$, using the bound as the base case.



\section{Chaining and Lemma Application}

There is a group of tactics based on the well-known notions of backward
and forward chaining.  The description of these tactics requires two
definitions.  For terms $A$ and $B$, $A$
is an {\em assumption of} $B$ if: $B$ is of the form {\tt $x$:$C$->$D$} (which
may be degenerate, {\em i.e.}, an implication) and $A$ is either identical to
$C$ or is an assumption of $D$; or $B$ is of the form {\tt $C$\#$D$} and $A$ is
an assumption of $C$ or of $D$.  $A$ is a {\em conclusion of} $B$ if: $A$ and
$B$ are identical; or $B$
is {\tt $x$:$C$->$D$} and $A$ is a
conclusion of $D$; or $B$ is {\tt $C$\#$D$} and $A$ is a conclusion
of $C$ or of $D$.  As assumption $A$ of $B$ is an antecedent in $B$ of a
conclusion $A'$ of $B$ if: $B$ is {\tt $x$:$C$->$D$} and either $A$ is $C$
and $A'$ is in $D$ or $A$ is an antecedent of $A'$ in $D$; or $B$ is {\tt
$C$\#$D$} and either $A$ is an antecedent of $A'$ in $C$ or $A$ is an
antecedent of $A'$ in $D$.

{\tt BackThruHypUsing $l$ $i$}.  Do one backward chaining step using hypothesis
$i$.  This uses {\tt match\_part\_in\_context} (described earlier) to match the
conclusion $C$ of the sequent against a conclusion of the type $A$ that is 
hypothesis $i$.  The terms in $l$ are used as additional instantiating terms.
If the match is successful, then the tactic generates as subgoals sequents with
the same hypothesis list, but with conclusions that are suitble
instantiations of assumptions of $A$
that are antecedents in $A$ of the subterm occurrence matched with $C$.

Several tactics are based on {\tt BackThruHypUsing}.

{\tt BackchainWith $T$}.  This is most concisely described by ML code:  
\begin{Numath}
\NuColumn{%
\Nu{\N{}let BackchainWith Tactic =}\\%
\Nu{\N{}Try Hypothesis THEN}\\%
\Nu{\N{}AtomizeConcl THENW}\\%
\Nu{\N{}( ApplyToAHyp (\mlambda{}i. BackThruHyp i THENM BackchainWith Tactic)}\\%
\Nu{\N{}  ORELSE Tactic}\\%
\Nu{\N{}).}%
}%
\end{Numath}%
{\tt Hypothesis} proves goals where the conclusion is identical to a hypothesis,
{\tt Atomize\-Concl} repeatedly applies introductions until the conclusion is not
a conjunction or function type, and {\tt ApplyToAHyp $T$}, for $T$ of type {\tt
int->tactic}, applies $T$ to each hypothesis in turn until success ({\em i.e.},
until an application does not fail).  A useful
instance of this tactic is {\tt BackchainWith Fail}, where {\tt Fail} is the
tactic which always fails.  This implements a kind of Horn-clause theorem
prover (based on depth-first search).  It is used by the tactic {\tt
Contradiction}, which attempts to show the hypothesis list to be contradictory
by asserting {\tt void} and backchaining.  Note that {\tt BackchainWith} does
not check for looping, although it would be easy to modify it so that it does
(in fact the current version does check).

There are two main ways to apply lemmas.  One is to apply them explicitly
using the tactics described below.  The other involves explicitly grouping
them according to similarity of usage.  The grouping is accomplished by using
a naming conventions, as in the association of typing lemmas to definitions,
or by using ML objects in the library to update global lists of tactics
and lemma names.  There are several tactics that refer to such lists; they
generally use these lists by trying to apply every tactic in the list or
every lemma named in the list.  For example, the tactics {\tt Autotactic},
{\tt Member}
and {\tt Inclusion}, described in a later section, each (as a last resort)
refer to an assooiated list.  Below we give some other examples.  This
technique is rather {\em ad hoc}, but it has turned out to be very useful in
practice.  

{\tt LemmaUsing \id{name} $l$}.  This is analogous to {\tt BackThruHyp},
except that a lemma ({\em i.e.}, previously proved theorem), instead of a
hypothesis, is
``backed through''.  The tactic {\tt Lemma} can be used when there is no term list
$l$.

{\tt FLemma \id{name} $l$}.  If the hypotheses whose
numbers appear in $l$ can be matched to some of the assumptions of
the
theorem named by \id{name}, then a subgoal is generated that has the same
conclusion and has as an additional hypothesis the smallest conclusion of the
named theorem such that all its antecedent assumptions were matched.  As in
{\tt match\_part\_in\_context}, types of quantified variables are used to
compute further instantiating terms.  A typical example of the use of {\tt
FLemma} is in the application of a transitivity property.  Suppose $r$ has been
proved, in a lemma named \id{foo}, to be transitive, {\em i.e.},
\begin{Numath}
\Nu{\mforall{}x,y,z:$A$. $r$(x,y) => $r$(y,z) => $r$(x,z).}
\end{Numath}%
If the first and
second hypotheses in some sequent are {\tt $r$($x$,$y$)} and {\tt $r$($y$,$z$)},
then an application of {\tt FLemma}, with arguments \id{foo} and {\tt [1;2]},
will generate a subgoal which has {\tt $r$($x$,$z$)} as a new hypothesis.



{\tt RewriteConclWithLemma \id{name}}.  If the named lemma has an equality as
a conclusion, then try to match the left side of the equality against a subterm
of the conclusion of the sequent; if successful, substitute for that subterm the
appropriate instance of the right side of the equality.  This may generate
non-trivial subgoals, other than the one resulting from the substitution, since
the equality asserted by the lemma may have preconditions.  This tactic has
several variants.

{\tt Decidable.} This just repeatedly applies tactics from a global list.
The members of the list are typically applications of {\tt Lemma} to the name
of a theorem that asserts the decidability of some proposition ({\em i.e.},
that has a conclusion of the form $P \vee \neg P$).  A related tactic is
{\tt Decide} which,
given $P$, produces subgoals for the cases $P$ and $\neg P$, calling {\tt
Decidable} on the subgoal $P \vee \neg P$.

{\tt Assume $P$}.  This is similar to {\tt Decide}, except that it deals
with goals of the form {\tt ?$P$}, where {\tt ?$P$} is defined to be {\tt
$P$|True}.  Types of this form are used to simulate ML-style failure in
Nuprl.  {\tt Assume} $P$, if successful, generates two subgoals, one where
$P$ is a new assumption, and one which is identical to the goal.  {\tt
Assume} looks for a previously defined function that either produces a proof
of $P$ or fails.  The two subgoals correspond to the two possible outcomes.

\section{A Type Constructor}

For the purpose of this section, define a {\em module} to be a function of the
form 
\begin{Numath}
\Nu{$\lambda{} x_1 \ldots{} x_m$. %
$y_1$:$B_1$ \# \ldots \# $y_{n-1}$:$B_{n-1}$ \# \{$y_n$:$B_n$|$C$\}.}   
\end{Numath}%
that is a member of a type
\begin{Numath}
\Nu{$x_1$:$A_1$ -> \ldots -> $x_m$:$A_m$ -> U$i$.}
\end{Numath}%
Many data types that arise in programming and in mathematics are modules by this
definition (which is nonstandard).  To help create and use such parameterized
data types,
there is a function {\tt create\_module}.  The inputs to this function are
the terms $A_i$, $B_j$, and $C$, a name $M$ for the module, names for Nuprl
functions for projecting components of members of the type, a library
position at
which to begin creating the associated library objects, 
a preferred name for a typical
element of the module, and the universe level $i$.

Executing the function creates library objects for the following:
\begin{itemize}
\item The definition of the module $M$.  This includes a definition object, and a
theorem from which the module is extracted.  The theorem is usually 
proved automatically.
\item Definitions of $n$ implicitly-polymorphic projection functions.  For any
members $a_1$, \ldots, $a_m$ of $A_1$, \ldots, $A_m$, respectively, the
$i^{\id{th}}$ projection function selects the $i^{\id{th}}$ component from an
$n$-tuple that is a member of {\tt $M$($a_1$,\ldots,$a_m$)}.  The theorems that
give
the types of the projections are generated, and usually proved, automatically.
\item A theorem that states that $C$, which can be thought of as an axiom about
members of the module, is true of the projections of any members of any {\tt
$M$($a_1$,\ldots,$a_m$)}.
\item Updating the {\tt tag\_redex} function (described above).  An ML object is
created that has the effect of advising certain tactics that perform
reduction to
consider an application of a defined projection to an $n$-tuple to be a
redex-contractum pair.
\item Updating the generic module eliminator.  An ML object is created that has
the effect of updating the {\tt ModE} tactic to take account of the new module.
This tactic can be applied to any hypothesis whose type is an instance ({\em
i.e.}, application) of a module that has been defined by {\tt create\_module}. 
The tactic works like {\tt GenProductE}, and chooses new
variable names that are based on the names of the projection functions.
\end{itemize}


\section{Derived Rules via Pattern Theorems}

Often one wishes to write a tactic that simply summarizes a pattern of
inference with a fixed structure.  In such cases, it is often easier to
partially prove a ``pattern theorem'' that contains syntactic variables which
get instantiated when the ``derived rule'' is applied.  Nuprl does not
directly support this kind of construction of derived rules, but it is
possible to simulate it.  The method of simulation is somewhat inelegant; it
is presented here because it is useful, and because it was used in the
formalization of Girard's Paradox (which is contained in the directory
\tid{nuprl/lib}).

In stating these pattern
theorems, atoms are used as
placeholders for irrelevant subterms.  The statement of this theorem is 
\begin{Numath}
\Nu{>> \mforall{}s:Term0. "G".}
\end{Numath}%

Immediately preceding a pattern theorem in the library should be a
corresponding ML object.  This ML object, when checked, should use the function
{\tt set\_\-d\_tactic\_\-args} to set certain global variables to contain the
arguments that are necessary for the proof of the pattern theorem.  
During the proof of the pattern theorem, a tactic needing arguments that
cannot be computed from the sequent it is applied to uses a special function to
access the global variables.  For example, a tactic may require terms to
instantiate a universal quantifier with; one of the global variables refers to
a list of terms from which the tactic may remove the needed number of terms by
using the function {\tt get\_term\_arg}.

The values given to the global variables by the ML object are just
placeholders for the actual
arguments that will be supplied by the tactic {\tt Pattern} which uses the
theorem as a pattern for
constructing another proof.  {\tt Pattern}, given
as input a set of actual arguments, assigns the arguments to the
appropriate variables, and then applies, in depth-first order, the rules that
occur in the pattern proof tree.  

Many of the derived rules that are defined with this mechanism resemble
elimination or introduction rules; that is, they analyze either a hypothesis
or the conclusion.  It is useful to encapsulate such tactics in a pair
of tactics {\tt DE} and {\tt DI} (also called {\tt DElim} and {\tt
DIntro}).  These tactics try to apply members of an associated
global list, continuing through the list until success.  The user can have
these lists updated from the library.



\section{Type Checking and the Autotactic}

\label{autotactic}

As mentioned earlier, the most important single tactic is the autotactic.  In
this section, the autotactic and related tactics are discussed.

{\tt Inclusion $i$}.  Attempt to prove that a term $t$ is in a type $T'$
assuming that hypothesis $i$ asserts that 
$t$ is a member of a type $T$.  This
tactic will make progress in the following situations.
\begin{itemize}
\item One of the members of a user-defined list of tactics succeeds.
\item $T$ and $T'$ are (or compute to) universes, with the level of $T$ not
greater than that of $T'$.
\item $T$ is a subtype of $T'$.
\item $T$ and $T'$ ``almost normalize'' to identical terms.  {\tt HypModComp},
described in Section~\ref{computation}, is used here.
\item $T$ and $T'$ are function types {\tt $x$:$A$->$B$} and {\tt
$x$:$A$->$B'$}, respectively, and {\tt Inclu\-sion} recursively
makes progress on the goal
\begin{Numath}
\Nu{$x$:$A$, $y$:$B$ >> $y$ in $B'$}
\end{Numath}%
\end{itemize}
The last case could be generalized to other type
constructors.   {\tt Inclu\-sion} is called
frequently, mostly by {\tt EqI} (see Section~\ref{derived}).  

{\tt PolyMember}.  Deal with membership goals {\tt $t$~in~$T$} where $t$ is an
instance of a definition that is of the second kind mentioned in
Section~\ref{definitions}.  The tactic uses matching to compute terms with which
to instantiate the typing lemma for the definition.  This results in a new
hypothesis {\tt $t$~in~$T'$}.  If $T$ and $T'$ are not identical, {\tt
Inclusion} is applied.  The subgoals usually produced by {\tt PolyMember} are to
show that the arguments of the definition instance are of the appropriate type.
{\tt
PolyMember} can also be applied when the conclusion is a binary equality with
both sides of the equality being instances of the same definition. In this case,
the subgoals will be to prove that the corresponding arguments are equal in the
appropriate types.

{\tt MemberI}.  Make ``one step'' of progress on a membership goal {\tt
$t$~in~$T$}.  What a ``step'' is depends on the outermost forms of $t$ and $T$.
Generally, the steps are conservative, and if the step that is determined to be
applicable fails, no other attempts are made.
For some cases, what step to take is obvious.  For example, if $t$ is {\tt
<$a$,$b$>}, and $T$ is {\tt $A$\#$B$}, then the step is to apply {\tt EqI} ({\em
i.e.}, to apply the appropriate Nuprl equality-intro rule).  In other cases,
there are several reasonable alternatives.  For example, if the goal is 
\begin{Numath}
\Nu{>> $f$($a$) in \{$x$:$A$|$B$\},}
\end{Numath}%
then {\tt EqI} might succeed.  But one can also proceed by analyzing the
set-type, getting subgoals
\begin{Numath}
\Nu{>> $f$($a$) in $A$   {\rm and}   >> $B$[$f$($a$)/$x$].}
\end{Numath}%
{\tt MemberI} takes the second alternative (assuming {\tt PolyMember} does not
apply).  However, if $T$ had not been explicitly a set type, but a term
that {\em computed}, via at least one definition expansion, to a set type,
then
the first alternative would have been taken.  This heuristic has worked well in
practice.  There are a few other simple heuristics used by {\tt MemberI}.  {\tt
MemberI} is another one of the tactics that can be updated by the user; the user
additions are tried first.  Also, as with {\tt PolyMember}, {\tt MemberI} can be
applied to binary equalities when the outermost forms of both sides are the
same.

{\tt Member}.  Reduce the conclusion (without expanding definitions), then
repeatedly apply {\tt MemberI} to membership goals where the member is not an
induction form.  The reason for stopping at induction forms is that it is
difficult to guess the types that are necessary to proceed.  Proceeding with bad
guesses can often lead to false subgoals.  
The conservative nature of {\tt Member} gives typechecking in Nuprl an
interactive character; most of typechecking can be handled automatically, but
occasionally the tactic will stop and require the user to make a
decision.  Fortunately, in practice {\tt Member} is able to automatically prove
almost all membership goals that most users would consider trivial.

{\tt Autotactic}.  The autotactic cycles through the following, stopping when
no further progress can be made.
\begin{itemize}
\item Decompose hypotheses that are conjunctions.
\item Remove all top level applications of the squash operator in the hypothesis
list.
\item Apply {\tt Member}.
\item Analyze set types in order to expose information for {\tt arith} and
then apply it.  This fails if {\tt arith} fails.
\item Analyze the conclusion ({\em i.e.}, apply an introduction step) if it is a
conjunction or a function-type.
\item Apply a member of a user-defined list.
\end{itemize}
(a second clause dealing with integer arithmetic is omitted).
Except for {\tt Member} and the last clause, all the clauses of the
autotactic are strongly
valid: if a goal is provable, then each subgoal generated by the clause is also
provable.  This property, the conservative nature of {\tt Member}, and the
fact that little backtracking is done, make it feasible to apply the autotactic
everywhere desired.  For example, users typically call it after almost every
top-level application 
of one of the tactics discussed in Section~\ref{derived}.

\section{An Example From Number Theory}

One of the first substantial developments of mathematics carried out in Nuprl
was a proof of the fundamental theorem of arithmetic.  The complete
self-contained library constructed for this theorem contains 113 objects, of
which 59 are definitions and 54 are theorems.  36 of the definitions are
common to most existing libraries; they are mostly definitions for logical
notions.  There are 15 definitions dealing with
basic list and integer relations and types, and 8 definitions which are 
particular to the development of the fundamental theorem of arithmetic.

Of the 54 theorems, 20 are associated with definitions in the way described
in Section~\ref{definitions}.
Most of the remaining theorems establish simple properties of the defined
objects.  The fundamental theorem of arithmetic is formalized
as the two Nuprl theorems
\begin{Numath}
\Nu{\mforall{}n:\{2..\}. \mexists{} l:PrimeFact where eval(l) = n}
\end{Numath}%
and
\begin{Numath}
\Nu{\mforall{} l1,l2:PrimeFact. eval(l1) = eval(l2) => l1 = l2 in Fact.}
\end{Numath}%
The first statement can be
read as: {\em for every integer $n$ greater than or equal to 2 there is a prime
factorization that multiplies out (or {\em evaluates}) to $n$}.  The second
can be read as: {\em any two prime factorizations that evaluate to the same
number are the
same factorizations}.  

The program extracted from the first theorem above is a function mapping
numbers to prime factorizations.  This program is mostly developed in the
proof of
the following lemma (the theorem is just an instantiation of the lemma).
%   THM >>  k:N.  n,i:{2..}
%              where in & n-ik &  d:{2..(i-1)}. (d|n). 
%           l:PrimeFact where eval(l) = n
\begin{Numath}
\NuColumn{%
\Nu{\mforall{}k:N. \mforall{}n,i:\{2..\} where i$\le$n \& n-i$\le$k \& %
\mforall{}d:\{2..(i-1)\}. $\neg$(d|n). }\\%
\Nu{    \mexists l:PrimeFact where eval(l) = n}%
}%
\end{Numath}%
This lemma is proved by induction on {\tt k} and suggests the algorithm used
to compute prime factorizations.  If we let $P(i,n)$ be 
\begin{Numath}
\Nu{$i$$\le$$n$ \& \mforall{}d:\{2..($i$-1)\}. $\neg$(d|$n$),}%
\end{Numath}%
$t(i,n)$ be {\tt $n$-$i$}, and $R(n)$ be 
\begin{Numath}
\Nu{\mexists{}l:PrimeFact where eval(l) = $n$,}%
\end{Numath}%
then the lemma is equivalent to (omitting some types)
\begin{Numath}
\Nu{\mforall{}k,n,i. $P(\tid{i},\tid{n})$ \& $t(\tid{i},\tid{n})$$\le$k %
=> $R(\tid{n})$.}% 
\end{Numath}%
If $P(i,n)$ and $t(i,n)=0$ then $n=i$ and $n$ is prime, so the base case of
the inductive proof of this lemma follows.
For the induction case,
assume $P(i,n)$.  If $t(i,n)=0$ we are done, otherwise we compute $n'$ and $i'$
such
that $t(i',n')<t(i,n)$ and $P(i',n')$, and use the induction hypothesis.  The
obvious choices for $i'$ and $n'$ are $i+1$ and $n$ in the case that $i$
does not divide $n$, and $i$ and $n/i$ otherwise.  In the first case $R(n)$
is proven by using the factorization given by $R(n')$; in the second, $i$ is
added to that factorization.  

It is possible to express as a theorem in Nuprl a general inductive schema
for problems that can be phrased, as in this case, in terms of an
``invariant'' $P$, a result assertion $R$ and a bounding function $t$
(although such a schema was not used in our proof of the fundamental theorem of
arithmetic).  One way of formulating such a schema is the
following.
% >> A,B:U1. P:(A#B)->U1. R:B->U1. t:(A#B)->Int.                       
%      x:A#B. P(x) => 0<t(x)                                              
%      & x:A#B. P(x) => R(x.2)                                            
%                         y:A#B. P(y) & t(y)<t(x) & (R(y.2) => R(x.2))   
%      & b:B. a:A. P(<a,b>)                                              
%      => b:B. R(b)                                                       
% \begin{Numath}
% \NuColumn{%
% \Nu{\N{}\mforall{}A,B:U1. \mforall{}P:(A\#B)->U1. \mforall{}R:B->U1. %
% \mforall{}t:(A\#B)->Int.}\\%
% \Nu{\N{}    \mforall{}x:A\#B. P(x) => 0<t(x)}\\%
% \Nu{\N{}  \& \mforall{}x:A\#B. P(x) => R(x.2)}\\%
% \Nu{\N{}                    \mvee \mexists{}y:A\#B. P(y) \& t(y)<t(x) \& %
% (R(y.2) => R(x.2))}\\%
% \Nu{\N{}  \& \mforall{}b:B. \mexists{}a:A. P(<a,b>)}\\%
% \Nu{\N{}  => \mforall{}b:B. R(b)}%
% }%
% \end{Numath}%
% >> A,B:U1. P:(A#B)->U1. R:B->U1. t:(A#B)->Int.              
%      x:A#B. t(x) = 0 & P(x) => R(x.2)                          
%      & x:A#B. P(x) & 0<t(x) =>                                 
%           y:A#B. P(y) & 0t(y)<t(x) & (R(y.2) => R(x.2))       
%      & b:B. a:A. 0t(<a,b>) & P(<a,b>)                        
%      => b:B. R(b)                                              
\begin{Numath}
\NuColumn{%
\Nu{\mforall{}A,B:U1. \mforall{}P:(A\#B)->U1. \mforall{}R:B->U1. %
\mforall{}t:(A\#B)->Int.}\\%
\Nu{\N  \mforall{}x:A\#B. t(x) = 0 \& P(x) => R(x.2) }\\%
\Nu{\N  \& \mforall{}x:A\#B. P(x) \& 0<t(x) =>}\\%
\Nu{\N      \mexists{}y:A\#B. P(y) \& 0\mleq{}t(y)<t(x) \& (R(y.2) => R(x.2))}\\%
\Nu{\N  \& \mforall{}b:B. \mexists{}a:A. 0\mleq{}t(<a,b>) \& P(<a,b>)}\\%
\Nu{\N  => \mforall{}b:B. R(b)}}
\end{Numath}%
To apply it to the lemma discussed above, we take $A$ and $B$ to be 
{\tt \{2..\}} and $P$, $R$ and $t$ to be suitable modifications of what was
given above. 

All of the proofs in the fundamental theorem of arithmetic library were
constructed using an early version of the tactic collection that was
described in
this chapter.  This early tactic collection was designed to be generally
applicable; none of the tactics were designed to be specifically useful in
number theory.

The total time required to complete the library was about forty
hours.  This includes all work relevant to the effort; in particular, it
includes the time spent on entering definitions, on informal planning, on
lemma discovery and aborted proof attempts, and on proving all the necessary
results dealing with the basic arithmetic operators and relations.   
The proofs contain a total of 879 refinement
steps, most of which were entered manually (some were automatically applied
by a primitive analogy tactic).

The fundamental theorem of arithmetic was chosen as a first major experiment
with Nuprl because it is widely recognized as a non-trivial theorem in
elementary number theory, because it has interesting computational content,
and because mechanical proofs of it have been performed in other systems.  It
took approximately 8 weeks of effort to prove the theorem in the PLCV
system.  PLCV is a natural deduction system for reasoning
about PL/C programs which has powerful built-in support for arithmetical and
propositional reasoning but no tactic mechanism or proof editor.  Boyer and
Moore also conducted a proof of the fundamental theorem of
arithmetic.  They do not say how long the effort took, but they do say in
their book that a correctness proof of a tautology checker required twelve
hours of effort, and that the fundamental theorem of arithmetic was without
doubt the hardest theorem proved so far using their system.  A comparison
with Boyer and Moore's proof is complicated by the fact that they started at
a lower level by
defining the integers and developing some elementary properties of numbers
which are incorporated in Nuprl's {\tt arith}  procedure.  On the other hand,
they also modified their system by adding heuristics for proving theorems in
number theory.


\section{Saddleback Search}

in this section we will look at a simple programming problem.  The
problem, whose solution has been called ``saddleback search'', is to
efficiently find the location of a given integer in a matrix of
integers whose rows and columns are sorted in non-decreasing order.
The complete solution, including planning, took one afternoon. 

The library that was constructed contains nine
objects in addition to a set of basic definitions that is shared by most
Nuprl
libraries.  The first of the nine is a simple definition for matrix application:
\begin{Numath}
\Nu{$B$($i$,$j$) \mequiv $B$($i$)($j$)}
\end{Numath}%
This is not precisely as the definition would appear if it were viewed on a
Nuprl screen; the parameters here are the italicized identifiers, whereas Nuprl
uses angle brackets ({\em e.g.}, {\tt <B>} instead of $B$).  The second two
objects (a theorem and a definition) together 
give a definition of the proposition that an
integer $x$ occurs
in an $m$ by $n$ matrix $B$ in a column between $i$ and $p$ and in a row between
$j$ and $q$: 
\begin{Numath}
\NuColumn{%
\Nu{$x$\mmin$B$\{$m$,$n$\}($i$:$p$, $j$:$q$)  \mequiv}\\%
\Nu{\mexists{}r:\{0..($m$-1)\}. \mexists{}s:\{0..($n$-1)\}. %
where $B$(r,s)=$x$ \& $i$$\leq$r$\leq$$p$ %
\&  $j$$\leq$s$\leq$$q$.}%
}%
\end{Numath}%
The next two objects in the library
define the property of an $m$ by $n$ matrix $B$ being sorted:
\begin{Numath}
\NuColumn{%
\Nu{sorted($B$)\{$m$,$n$\}  \mequiv}\\%
\Nu{\mforall{}j:\{0..($n$-1)\}. \mforall{}i,p:\{0..($m$-1)\}. %
i<p  =>  $B$(i,j)$\leq$$B$(p,j) \&}\\%
\Nu{\mforall{}i:\{0..($m$-1)\}. \mforall{}j,q:\{0..($n$-1)\}. j<q%
  =>  $B$(i,j)$\leq$$B$(i,q).}%
}%
\end{Numath}%
Following this are two ML objects which contain the definitions of two
simple tactics called {\tt RowSorted} and {\tt ColSorted}
which are used to apply the fact that a matrix is sorted.

The final two objects in the library are
the main theorem and a lemma.  The statement of the main theorem is
% >> m,n:N+.  B:({0..(m-1)}->{0..(n-1)}->Int) where sorted(B{m,n}).  x:Int. 
%          x  B{m,n}(0:(m-1), 0:(n-1))    (x  B{m,n}(0:(m-1), 0:(n-1)))
\begin{Numath}
\NuColumn{%
\Nu{\mforall{}m,n:N+. \mforall{}B:(\{0..(m-1)\}->\{0..(n-1)\}->Int) %
where sorted(B\{m,n\}).}\\%
\Nu{\N{} \mforall{}x:Int. x\mmin{}B\{m,n\}(0:(m-1),0:(n-1)) %
\mvee{} $\neg$(x\mmin{}B\{m,n\}(0:(m-1),0:(n-1)))}%
}%
\end{Numath}%
The program extracted from a proof of this theorem is a function that when
given inputs of the appropriate type produces an indication of whether or not
$x$ occurs in the matrix $B$ together with, in the former case, the pair of
integers which give the location of $x$.

The algorithm implemented by our proof proceeds as follows.  Start at the
lower left corner of the matrix ({\em i.e.}, at position $\Tuple{m-1,0}$).  If
$x$ is equal to
the matrix element $b$ at this position, then we are done.  If $x<b$ 
then since the matrix is sorted $x$ must be smaller than everything in the
last row, so we can throw away the last row and repeat the procedure with the
smaller
matrix.  Similarly, if $x>b$, then we can discard the first column and
repeat. This algorithm has a simple recursive structure, where a solution for an
upper-right block of the matrix can be found in terms of a solution for a
smaller block obtained by deleting the bottom row or the leftmost column.  
This is the motivation for the lemma:
% >> m,n:N+. B:({0..(m-1)}->{0..(n-1)}->Int) where sorted(B{m,n}).  
%    x:Int. k:N. i:{0..(m-1)}. j:{0..(n-1)} where i+(n-j) = k. 
%         x  B{m,n}(0:i, j:(n-1))    (x  B{m,n}(0:i, j:(n-1)))
\begin{Numath}
\NuColumn{%
\Nu{\mforall{}m,n:N+. \mforall{}B:(\{0..(m-1)\}->\{0..(n-1)\}->Int) %
where sorted(B\{m,n\}).  }\\%
\Nu{\N{}  \mforall{}x:Int.\mforall{}k:N. \mforall{}i:\{0..(m-1)\}. %
\mforall{}j:\{0..(n-1)\} where i+(n-j) = k. }\\%
\Nu{\N{}     x \mmin B\{m,n\}(0:i, j:(n-1))  \mvee  %
$\neg$(x \mmin B\{m,n\}(0:i, j:(n-1)))}%
}%
\end{Numath}%
This can be interpreted as asserting that for every $k$, we can find whether or
not $x$ occurs in any upper-right block of $B$ that has a total of $k-1$ rows and
columns.  


The proofs that contain the development of the saddleback search algorithm are
short enough to permit a complete presentation.  
The proof of the main lemma is by induction on $k$, and the first step is shown in
Figure~\ref{FirstStepAgain}.
\begin{RuledFigure}
\begin{Screen}{1}{\SnapshotSize}
\N{}* top  \\{}
\N{}>> \mforall{}m,n:N+. \mforall{}B:(\{0..(m-1)\}->\{0..(n-1)\}->Int) where sorted(B\{m,n\}).   \\{}
\N{}     \mforall{}x:Int. \mforall{}k:N. \mforall{}i:\{0..(m-1)\}. \mforall{}j:\{0..(n-1)\} where i+(n-j) = k.  \\{}
\N{}          x \(\epsilon{}\) B\{m,n\}(0:i, j:(n-1))  \mvee{}  \mneg{}(x \(\epsilon{}\) B\{m,n\}(0:i, j:(n-1))) \\{}
\N{} \\{}
\N{}BY (OnVar `k` (NonNegInd `l`) ...) \\{}
\N{} \\{}
\N{}1* 1. m: N+ \\{}
\N{}   2. n: N+ \\{}
\N{}   3. B: \{0..(m-1)\}->\{0..(n-1)\}->Int \\{}
\N{}   4. x: Int \\{}
\N{}   [5]. sorted(B\{m,n\}) \\{}
\N{}   6. i: \{0..(m-1)\} \\{}
\N{}   7. j: \{0..(n-1)\} \\{}
\N{}   [8]. i+(n-j)=0  \\{}
\N{}   >> x \(\epsilon{}\) B\{m,n\}(0:i, j:(n-1))  \mvee{}  \mneg{}(x \(\epsilon{}\) B\{m,n\}(0:i, j:(n-1))) \\{}
\N{} \\{}
\N{}2* 1. m: N+ \\{}
\N{}   2. n: N+ \\{}
\N{}   3. B: \{0..(m-1)\}->\{0..(n-1)\}->Int \\{}
\N{}   4. x: Int \\{}
\N{}   5. l: int \\{}
\N{}   6. 0<l \\{}
\N{}   7. \mforall{}i:\{0..(m-1)\}. \mforall{}j:\{0..(n-1)\} where i+(n-j)=l-1.  \\{}
\N{}         x \(\epsilon{}\) B\{m,n\}(0:i, j:(n-1)) \mvee{}  \mneg{}(x \(\epsilon{}\) B\{m,n\}(0:i, j:(n-1))) \\{}
\N{}   [8]. sorted(B\{m,n\}) \\{}
\N{}   9. i: \{0..(m-1)\} \\{}
\N{}   10. j: \{0..(n-1)\} \\{}
\N{}   [11]. i+(n-j)=l in Int \\{}
\N{}   >> x \(\epsilon{}\) B\{m,n\}(0:i, j:(n-1))  \mvee{}  \mneg{}(x \(\epsilon{}\) B\{m,n\}(0:i, j:(n-1)))    
\end{Screen}%
\caption{By induction.}
\label{FirstStepAgain} 
\end{RuledFigure}
The tactic used here performs induction on $k$, using $l$ as a new variable for
the induction step.  The square brackets around some of the hypotheses in the
subgoals indicate that the corresponding hypotheses are {\em hidden}.  A
hidden hypothesis cannot be used in any way, and remains hidden through
further rule applications until the conclusion has a computationally trivial
form ({\em e.g.}, when it is an equality).

The first subgoal is proved in two steps.  The first step, shown in 
Figure~\ref{SetUpBaseContr}, is to
assert that
the hypotheses are contradictory and to expand types which are subsets.  
The display of this step in Nuprl
would show as part of the main goal the hypotheses numbered {\tt 1} to {\tt 8}
shown in Figure~\ref{FirstStepAgain}. In
the interest of compactness, these have been manually elided (replaced by
the string ``{\tt{}...}'').  
In addition, the system itself suppresses the display in subgoals
of hypotheses that also appear in the goal.
\begin{RuledFigure}
\begin{Screen}{1}{\SnapshotSize}
\N{}* top 1 \\{}
\N{}... \\{}
\N{}>> x \(\epsilon{}\) B\{m,n\}(0:i, j:(n-1))  \mvee{}  \mneg{}(x \(\epsilon{}\) B\{m,n\}(0:i, j:(n-1))) \\{}
\N{} \\{}
\N{}BY (Assert 'False' ...) THEN (Unsetify ...) \\{}
\N{}     \\{}
\N{}1* 1. m: Int \\{}
\N{}   2. 0<m \\{}
\N{}   3. n: Int \\{}
\N{}   4. 0<n \\{}
\N{}   8. i: Int \\{}
\N{}   9. j: Int \\{}
\N{}   11. 0 \mleq{} j \\{}
\N{}   12. j \mleq{} n-1 \\{}
\N{}   13. 0 \mleq{} i \\{}
\N{}   14. i \mleq{} m-1 \\{}
\N{}   >> False    
\end{Screen}%
\caption{The base case is contradictory.}
\label{SetUpBaseContr} 
\end{RuledFigure}
The second step, shown in Figure~\ref{BaseCaseContr}, is to show that a
contradiction can be inferred from the known equalities and inequalities.
This is accomplished by using tactics based on the monotonicity rule to add
hypothesis {\tt 10} to hypothesis {\tt 13}, and then to add $n$ to both sides
of {\tt 13}.
\begin{RuledFigure}
\begin{Screen}{1}{\SnapshotSize}
\N{}* top 1 1 \\{}
\N{}1. m: Int \\{}
\N{}2. 0<m \\{}
\N{}3. n: Int \\{}
\N{}4. 0<n \\{}
\N{}5. B: \{0..(m-1)\}->\{0..(n-1)\}->Int \\{}
\N{}6. x: Int \\{}
\N{}7. sorted(B\{m,n\}) \\{}
\N{}8. i: Int \\{}
\N{}9. j: Int \\{}
\N{}10. i+(n-j)=0 in Int \\{}
\N{}11. 0 \mleq{} j \\{}
\N{}12. j \mleq{} n-1 \\{}
\N{}13. 0 \mleq{} i \\{}
\N{}14. i \mleq{} m-1 \\{}
\N{}>> False \\{}
\N{} \\{}
\N{}BY (Mono 10 `+` 12 THEN MonoWithR 13 `+` 'n = n' ...) 
\end{Screen}%
\caption{Contradiction follows from known inequalities.}
\label{BaseCaseContr} 
\end{RuledFigure}

The first step of the proof of the induction step (the second subgoal
in Figure~\ref{FirstStepAgain}) is shown in
Figure~\ref{MainCaseSplitAgain}.  
In this step, we are considering the block of the
matrix that extends from the top to row $i$ and from column $j$ to the right.  
The induction hypothesis (hypothesis {\tt 7} in the second subgoal in
Figure~\ref{FirstStepAgain})
asserts the decidability of membership of $x$ in smaller blocks of $B$.
As in the informal description of the algorithm, we do a case analysis
on the relation between $x$ and $B(i,j)$.
\begin{RuledFigure}
\begin{Screen}{1}{\SnapshotSize}
\N{}* top 2 \\{}
\N{}... \\{}
\N{}>> x \(\epsilon{}\) B\{m,n\}(0:i, j:(n-1))  \mvee{}  \mneg{}(x \(\epsilon{}\) B\{m,n\}(0:i, j:(n-1))) \\{}
\N{} \\{}
\N{}BY (Cases ['B(i,j)<x'; 'B(i,j)=x'; 'x<B(i,j)'] ...) \\{}
\N{}    \\{}
\N{}1* 12. B(i,j)<x \\{}
\N{}   >> x \(\epsilon{}\) B\{m,n\}(0:i, j:(n-1))  \mvee{}  \mneg{}(x \(\epsilon{}\) B\{m,n\}(0:i, j:(n-1))) \\{}
\N{} \\{}
\N{}2* 12. B(i,j)=x \\{}
\N{}   >> x \(\epsilon{}\) B\{m,n\}(0:i, j:(n-1))  \mvee{}  \mneg{}(x \(\epsilon{}\) B\{m,n\}(0:i, j:(n-1))) \\{}
\N{} \\{}
\N{}3* 12. x<B(i,j) \\{}
\N{}   >> x \(\epsilon{}\) B\{m,n\}(0:i, j:(n-1))  \mvee{}  \mneg{}(x \(\epsilon{}\) B\{m,n\}(0:i, j:(n-1)))       
\end{Screen}%
\caption{The main case split.}
\label{MainCaseSplitAgain} 
\end{RuledFigure}%
The case where $B(i,j)=x$ is easy; we can justify the left disjunct by
providing the coordinates where $x$ occurs.  This step is shown in
Figure~\ref{TrivCase}.
\begin{RuledFigure}
\begin{Screen}{1}{\SnapshotSize}
\N{}* top 2 2 \\{}
\N{}... \\{}
\N{}12. B(i,j) = x \\{}
\N{}>> x \(\epsilon{}\) B\{m,n\}(0:i, j:(n-1))  \mvee{}  \mneg{}(x \(\epsilon{}\) B\{m,n\}(0:i, j:(n-1))) \\{}
\N{} \\{}
\N{}BY (ILeft THENW ITerms ['i';'j'] ...) 
\end{Screen}%
\caption{In this case, $x$ has been located.}
\label{TrivCase}
\end{RuledFigure}%
For the case where $B(i,j)<x$, we want to discard column
$j$ and use the induction hypothesis.  There is no point doing this if
column $j$ is the last column, however, since in that case $x$ does not
occur in the block being considered.  Hence we perform the case analysis shown
in Figure~\ref{LastColTest}.
\begin{RuledFigure}
\begin{Screen}{1}{\SnapshotSize}
\N{}* top 2 1 \\{}
\N{}... \\{}
\N{}12. B(i,j)<x \\{}
\N{}>> x \(\epsilon{}\) B\{m,n\}(0:i, j:(n-1))  \mvee{}  \mneg{}(x \(\epsilon{}\) B\{m,n\}(0:i, j:(n-1))) \\{}
\N{} \\{}
\N{}BY (Cases ['j<n-1'; 'j = n-1'] ...) \\{}
\N{}    \\{}
\N{}1* 13. j<n-1 \\{}
\N{}   >> x \(\epsilon{}\) B\{m,n\}(0:i, j:(n-1))  \mvee{}  \mneg{}(x \(\epsilon{}\) B\{m,n\}(0:i, j:(n-1))) \\{}
\N{} \\{}
\N{}2* 13. j = n-1 \\{}
\N{}   >> x \(\epsilon{}\) B\{m,n\}(0:i, j:(n-1))  \mvee{}  \mneg{}(x \(\epsilon{}\) B\{m,n\}(0:i, j:(n-1)))    
\end{Screen}%
\caption{Is the column to be discarded the last one?}
\label{LastColTest} 
\end{RuledFigure}

Consider first the case where $j<n-1$.  As shown in Figure~\ref{UseIndHyp}, we
apply the induction hypothesis to the smaller block of $B$ obtained by removing
column $j$.
\begin{RuledFigure}
\begin{Screen}{1}{\SnapshotSize}
\N{}* top 2 1 1 \\{}
\N{}... \\{}
\N{}7. \mforall{}i:\{0..(m-1)\}. \mforall{}j:\{0..(n-1)\} where i+(n-j)=l-1.  \\{}
\N{}      x \(\epsilon{}\) B\{m,n\}(0:i, j:(n-1)) \mvee{}  \mneg{}(x \(\epsilon{}\) B\{m,n\}(0:i, j:(n-1))) \\{}
\N{}... \\{}
\N{}12. B(i,j)<x \\{}
\N{}13. j<n-1 \\{}
\N{}>> x \(\epsilon{}\) B\{m,n\}(0:i, j:(n-1))  \mvee{}  \mneg{}(x \(\epsilon{}\) B\{m,n\}(0:i, j:(n-1))) \\{}
\N{} \\{}
\N{}BY (InstHyp ['i'; 'j+1'] 7 ...) \\{}
\N{}    \\{}
\N{}1* 14. x \(\epsilon{}\) B\{m,n\}(0:i, j+1:(n-1)) \mvee{}  \mneg{}(x \(\epsilon{}\) B\{m,n\}(0:i, j+1:(n-1))) \\{}
\N{}   >> x \(\epsilon{}\) B\{m,n\}(0:i, j:(n-1))  \mvee{}  \mneg{}(x \(\epsilon{}\) B\{m,n\}(0:i, j:(n-1)))    
\end{Screen}%
\caption{Try to find $x$ in the smaller block of $B$ obtained by removing column $j$.}
\label{UseIndHyp} 
\end{RuledFigure}%
Now, $x$ is in the larger block just in case it is in the smaller block
(Figure~\ref{AnalyseIndHyp}).
\begin{RuledFigure}
\begin{Screen}{1}{\SnapshotSize}
\N{}* top 2 1 1 1 \\{}
\N{}... \\{}
\N{}12. B(i,j)<x \\{}
\N{}13. j<n-1 \\{}
\N{}14. x \(\epsilon{}\) B\{m,n\}(0:i, j+1:(n-1)) \mvee{}  \mneg{}(x \(\epsilon{}\) B\{m,n\}(0:i, j+1:(n-1))) \\{}
\N{}>> x \(\epsilon{}\) B\{m,n\}(0:i, j:(n-1))  \mvee{}  \mneg{}(x \(\epsilon{}\) B\{m,n\}(0:i, j:(n-1))) \\{}
\N{} \\{}
\N{}BY (OnLastHyp CaseHyp THENL [ILeft;IRight] ...) \\{}
\N{}    \\{}
\N{}1* 14. x \(\epsilon{}\) B\{m,n\}(0:i, j+1:(n-1)) \\{}
\N{}   >> x \(\epsilon{}\) B\{m,n\}(0:i, j:(n-1)) \\{}
\N{} \\{}
\N{}2* 14. \mneg{}(x \(\epsilon{}\) B\{m,n\}(0:i, j+1:(n-1))) \\{}
\N{}   15. x \(\epsilon{}\) B\{m,n\}(0:i, j:(n-1)) \\{}
\N{}   >> void    
\end{Screen}%
\caption{$x$ occurs if and only if it occurs in the smaller block.}
\label{AnalyseIndHyp} 
\end{RuledFigure}%
The first case is shown in Figure~\ref{FoundInSmaller}.  If $x$ is in the
smaller block, say at
position $\Tuple{r,s}$, then it must also be in the larger block at
position $\Tuple{r,s}$.
\begin{RuledFigure}
\begin{Screen}{1}{\SnapshotSize}
\N{}* top 2 1 1 1 1 \\{}
\N{}... \\{}
\N{}12. B(i,j)<x \\{}
\N{}13. j<n-1 \\{}
\N{}14. x \(\epsilon{}\) B\{m,n\}(0:i, j+1:(n-1)) \\{}
\N{}>> x \(\epsilon{}\) B\{m,n\}(0:i, j:(n-1)) \\{}
\N{} \\{}
\N{}BY (OnLastHyp (CHThen (RepeatProductE ``r s``))  \\{}
\N{}    THEN ITerms ['r'; 's'] ...)    
\end{Screen}%
\caption{If $x$ occurs in the smaller block, we're done.}
\label{FoundInSmaller} 
\end{RuledFigure}%
The case where $x$ is not in the smaller block is handled in
Figure~\ref{Gleep}.  We are assuming that $x$ must occur in the larger block,
and need to obtain a contradiction.
Let $r$ and $s$ be
the position of $x$ in the larger block.  
We need to know whether $\Tuple{r,s}$ is
in column $j$ or in the smaller block, so we do the case analysis shown in 
the figure.
\begin{RuledFigure}
\begin{Screen}{1}{\SnapshotSize}
\N{}* top 2 1 1 1 2 \\{}
\N{}... \\{}
\N{}12. B(i,j)<x \\{}
\N{}13. j<n-1 \\{}
\N{}14. \mneg{}(x \(\epsilon{}\) B\{m,n\}(0:i, j+1:(n-1))) \\{}
\N{}15. x \(\epsilon{}\) B\{m,n\}(0:i, j:(n-1)) \\{}
\N{}>> void \\{}
\N{} \\{}
\N{}BY (OnLastHyp (CHThen (RepeatProductE ``r s``))  \\{}
\N{}    THEN Cases ['s = j'; 'j<s'] ...) \\{}
\N{}    \\{}
\N{}1* 15. r: \{0..(m-1)\} \\{}
\N{}   16. s: \{0..(n-1)\} \\{}
\N{}   17. s = j \\{}
\N{}   18. j \mleq{} s  \\{}
\N{}   19. s \mleq{} n-1 \\{}
\N{}   20. B(r,s)  =  x \\{}
\N{}   21. 0 \mleq{} r \\{}
\N{}   22. r \mleq{} i \\{}
\N{}   >> void \\{}
\N{} \\{}
\N{}2* 15. r: \{0..(m-1)\} \\{}
\N{}   16. s: \{0..(n-1)\} \\{}
\N{}   17. j<s \\{}
\N{}   18. j \mleq{} s  \\{}
\N{}   19. s \mleq{} n-1 \\{}
\N{}   20. B(r,s)  =  x \\{}
\N{}   21. 0 \mleq{} r \\{}
\N{}   22. r \mleq{} i \\{}
\N{}   >> void    
\end{Screen}%
\caption{$x$ does not occur in the smaller block.  
Suppose it occurs at $\Tuple{r,s}$ the larger block.  
We need to know whether $\Tuple{r,s}$ is in column $j$.}
\label{Gleep} 
\end{RuledFigure}
The first case, where $s=j$, is shown in Figure~\ref{ColSortedContr}; a
contradiction 
follows from the fact that column $j$ is sorted (and so $B(r,j) \leq B(i,j)$).  
\begin{RuledFigure}
\begin{Screen}{1}{\SnapshotSize}
\N{}* top 2 1 1 1 2 1 \\{}
\N{}... \\{}
\N{}8. sorted(B\{m,n\}) \\{}
\N{}... \\{}
\N{}12. B(i,j)<x \\{}
\N{}13. j<n-1 \\{}
\N{}14. \mneg{}(x \(\epsilon{}\) B\{m,n\}(0:i, j+1:(n-1))) \\{}
\N{}15. r: \{0..(m-1)\} \\{}
\N{}16. s: \{0..(n-1)\} \\{}
\N{}17. s = j \\{}
\N{}18. j \mleq{} s  \\{}
\N{}19. s \mleq{} n-1 \\{}
\N{}20. B(r,s)  =  x \\{}
\N{}21. 0 \mleq{} r \\{}
\N{}22. r \mleq{} i \\{}
\N{}>> void \\{}
\N{} \\{}
\N{}BY (ColSorted 'j' 'r' 'i' 8 ...) 
\end{Screen}%
\caption{The contradiction follows from the fact that column $j$ is 
sorted.}
\label{ColSortedContr} 
\end{RuledFigure}%
In the second case, shown in Figure~\ref{SmallerBlockContr}, we get a
contradiction to hypothesis {\tt 14} by showing that $x$ does occur in
the smaller block. 
\begin{RuledFigure}
\begin{Screen}{1}{\SnapshotSize}
\N{}* top 2 1 1 1 2 2 \\{}
\N{}... \\{}
\N{}12. B(i,j)<x \\{}
\N{}13. j<n-1 \\{}
\N{}14. \mneg{}(x \(\epsilon{}\) B\{m,n\}(0:i, j+1:(n-1))) \\{}
\N{}15. r: \{0..(m-1)\} \\{}
\N{}16. s: \{0..(n-1)\} \\{}
\N{}17. j<s \\{}
\N{}18. j \mleq{} s  \\{}
\N{}19. s \mleq{} n-1 \\{}
\N{}20. B(r,s)  =  x \\{}
\N{}21. 0 \mleq{} r \\{}
\N{}22. r \mleq{} i \\{}
\N{}>> void \\{}
\N{} \\{}
\N{}BY (E 14 THENO ITerms ['r';'s'] ...) 
\end{Screen}%
\caption{$\Tuple{r,s}$ must be in the smaller block, contradicting 14.}
\label{SmallerBlockContr} 
\end{RuledFigure}
This finishes the proof of the case where $j<n-1$.

In the case where $j=n-1$, we prove (Figure~\ref{AtLastColContr})
that $x$ does not occur by contradiction.
If it occurred at position $\Tuple{r,s}$, say, then the fact that column $j$ is
column sorted would give us $B(r,j) \leq B(i,j)$, from which a
contradiction is immediate.
\begin{RuledFigure}
\begin{Screen}{1}{\SnapshotSize}
\N{}* top 2 1 2 \\{}
\N{}... \\{}
\N{}8. sorted(B\{m,n\}) \\{}
\N{}... \\{}
\N{}12. B(i,j)<x \\{}
\N{}13. j = n-1 \\{}
\N{}>> x \(\epsilon{}\) B\{m,n\}(0:i, j:(n-1))  \mvee{}  \mneg{}(x \(\epsilon{}\) B\{m,n\}(0:i, j:(n-1))) \\{}
\N{} \\{}
\N{}BY (IRight ...)  \\{}
\N{}   THEN (OnLastHyp (CHThen (RepeatProductE ``r s``)) ...) \\{}
\N{}   THEN (ColSorted 'j' 'r' 'i' 8 ...) 
\end{Screen}%
\caption{When $j$ is the last column of $B$, $x$ does not occur,
since if it did, 
the fact that column $j$ is sorted would be contradicted.}
\label{AtLastColContr} 
\end{RuledFigure}

This completes the proof in the case that $B(i,j)<x$.  The case $x<B(i,j)$ is
analogous, so it will not be discussed.  Except for this analogous branch, the
figures we have given cover the entire Nuprl proof of the lemma.  

The main theorem follows in two steps from the lemma.  The first step, shown in
Figure~\ref{MainThm}, is to instantiate the lemma using $m+n-1$, $m-1$, and $0$
for $k$, $i$, and $j$ respectively.  The final step (Figure~\ref{MainThmContr})
is to remove $n$ from the inequality in hypothesis {\tt 6}.
\begin{RuledFigure}
\begin{Screen}{1}{\SnapshotSize}
\N{}* top  \\{}
\N{}>> \mforall{}m,n:N+. \mforall{}B:(\{0..(m-1)\}->\{0..(n-1)\}->Int) %
where sorted(B\{m,n\}). \mforall{}x:Int. \\{}
\N{}      x \(\epsilon{}\) B\{m,n\}(0:(m-1), 0:(n-1))  %
\mvee{}  \mneg{}(x \(\epsilon{}\) B\{m,n\}(0:(m-1), 0:(n-1))) \\{}
\N{} \\{}
\N{}BY (Id ...) THEN  \\{}
\N{}   (LemmaUsing `saddleback\_lemma` ['m+n-1'; 'm-1'; '0'] ...) \\{}
\N{} \\{}
\N{}1* 1. m: N+ \\{}
\N{}   2. n: N+ \\{}
\N{}   3. B: \{0..(m-1)\}->\{0..(n-1)\}->Int \\{}
\N{}   4. sorted(B\{m,n\}) \\{}
\N{}   5. x: Int \\{}
\N{}   6. m+n-1<0 \\{}
\N{}   >> void    
\end{Screen}%
\caption{Use the main lemma, with $m+n-1$ for the bound, $m-1$ for starting 
row $i$, and $0$ for the starting column $j$.}
\label{MainThm} 
\end{RuledFigure}
\begin{RuledFigure}
\begin{Screen}{1}{\SnapshotSize}
\N{}* top 1 \\{}
\N{}1. m: N+ \\{}
\N{}2. n: N+ \\{}
\N{}3. B: \{0..(m-1)\}->\{0..(n-1)\}->Int \\{}
\N{}4. sorted(B\{m,n\}) \\{}
\N{}5. x: Int \\{}
\N{}6. m+n-1<0 \\{}
\N{}>> void \\{}
\N{} \\{}
\N{}BY (MonoWithR 6 `-` '0<n' ...) 
\end{Screen}%
\caption{Subtract {\tt 0<n} from hypothesis 6, getting {\tt m-1<0},
contradicting {\tt m>0}.} 
\label{MainThmContr} 
\end{RuledFigure}

For completeness we show in Figure~\ref{ColSorted} one of the two ML objects
in the lbirary.  This object defines the tactic {\tt ColSorted} which was
used in the steps shown in Figures~\ref{ColSortedContr}
and~\ref{AtLastColContr}.  This tactic was written to shorten the proof, and
its function is to instantiate the property {\tt sorted(B\{m,n\})} on two
positions within a single column.  The second ML object defines an analogous
tactic {\tt RowSorted}.
\begin{RuledFigure}
\begin{Screen}{1}{\SnapshotSize}
\N{}  let ColSorted col row1 row2 i p =\\{}
\N{}    let n = number\_of\_hyps p  in\\{}
\N{}    ( CH i \\{}
\N{}      THEN (E i)\\{}
\N{}      THEN InstantiateHyp [col; row1; row2] (n+1)\\{}
\N{}      THEN Thinning [n+1; n+2]\\{}
\N{}    ) p\\{}
\N{}  ;; 
\end{Screen}%
\caption{One of two simple tactics.}
\label{ColSorted} 
\end{RuledFigure}
