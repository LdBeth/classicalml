\chapter{The Proof Editor}

{}Nuprl proofs are sequent{} proofs organized
in a refinement style.
In other words they are trees where each node contains a {\em goal}{}
and a {\em rule{} application}.
The children of a node are the {\em subgoals}{}
of their parent; in order to prove the parent it is sufficient to prove
the children.
(Nuprl automatically generates the children, given their parent and the rule
to be applied; it is never necessary to type them in.)
Not all goals have subgoals; for example, a goal proved by {\tt hyp}
does not.
Also, an {\em incomplete} goal (one whose rule is bad or missing) has no
subgoals.

In what follows the {\em main goal} ({\em the} goal for short)
is the goal of the current node;
{\em a} goal is either the main goal or one of its subgoals.

\section{Basics of the Proof Editor}

The proof{} editor, {\em red}{} (for
{\em refinement{} editor}),
is invoked when the {\tt view} command is given on a theorem object.
The proof editor window always displays the {\em current node} of the proof.
When the editor is first entered the current node is the root of the tree;
in general, the current node can be changed by going into one of its subgoals
or by going to its parent.
For convenience there are also ways to jump to the next unproved leaf
and to the root of the theorem.

\begin{figure}
\hrule
\begin{center}
\begin{tabular}{|l|}\hline
\tt EDIT THM t \\ \hline
\verb`* top 1 2                         `\\
\verb`1. a:int                                                `\\
\verb`>> b:int # ((a=b in int) | ((a=b in int) -> void)) in U1`\\
\verb`                                                        `\\
\verb`BY intro                                                `\\
\verb`                                                        `\\
\verb`1* >> int in U1                                       `\\
\verb`                                                        `\\
\verb`2* 2. b:int                                             `\\
\verb`   >> ((a=b in int) | ((a=b in int) -> void))  in U1    `\\
\nothing\\ \hline  
\end{tabular}
\end{center}
\caption{A Sample Proof Editor Window}
\vspace{2pt}
\hrule
\label{ch7 - A sample proof editor window}
\end{figure}

Figure \ref{ch7 - A sample proof editor window}
shows a sample proof editor{} window.
The numbers below refer to the lines of text in the window, starting
with the line {\tt EDIT THM t}.
%
\begin {enumerate}
\item
    The header{} line names the theorem being edited, {\tt t}
    in this case.

\item
    The {\em map}{}, {\tt top 1 2},
    gives the path from the root to the current node.
    (If the path is too long to fit on one line it is truncated on the left.)
    The displayed node is the second child of the first child of the root.

\item The goal's first (and only) hypothesis{}, {\tt a:int},
    appears under the map.

\item
    The conclusion{} of the goal is the formula to be proved,
    given the hypotheses.
    The conclusion is shown to the right of the the sequent arrow, {\tt >>}.
    In the example it is
\begin{verbatim}
      b:int # ((a=b in int)|((a=b in int) -> void)) in U1.
\end{verbatim}

\item
    The {\em rule}{} specifies how to refine the goal.
    In the example the {\tt intro} rule is used to specify
    {\em product introduction}, since the main connective of the conclusion is
    the product operator, {\tt \#}.

\item
    The first subgoal has one hypothesis, {\tt a:int}, which is not
    displayed since it already occurs as a hypothesis of the main goal, and
    the conclusion {\tt int in U1}.

\item
    The second subgoal has two hypotheses, {\tt a:int} and {\tt b:int}, the
    first of which is not displayed,
    and the conclusion
\begin{verbatim}
        ((a=b in int) | ((a=b in int) -> void)) in U1.
\end{verbatim}
\end {enumerate}

Notice that lines (2), (6) and (7) have asterisks, which serve as
{\em status{} indicators}.
The asterisk in line (2) means that this entire subgoal has the status
{\em complete};
similarly, the asterisks in lines (6) and (7) mean that the two subgoals
also are {\em complete}.
Other status indicators are {\tt \#} (incomplete), {\tt -} (bad), and {\tt ?}
(raw).

\section{Motion within a Proof Node}

{}{}We first make the following definitions.
The proof{} cursor is {\em in} a goal if it
points to a hypothesis or the
conclusion;
it is {\em at} a goal if it points to the first part of the goal.
If the cursor is not in a goal, it is {\em at} the rule of the current node
or {\em in} or {\em at} a subgoal.

The cursor can be moved around within
the current proof node in the following ways:
\begin {description}
\item[\UP{}, \DOWN{}]{} \hfill \break
\noindent
    Moves the cursor up or down to the next or previous minor stopping point.
    A minor stopping point is an assumption, a conclusion or the rule.

\item[\LONG{} \UP{}, \LONG{} \DOWN{}]{} \hfill \break
\noindent
    Moves the cursor up or down to the next or previous major stopping point.
    A major stopping point is a subgoal, a rule or the goal.

\item[\LONG{} \LONG{} \UP{}]{} \hfill \break
\noindent
    Moves the cursor to the goal.

\item[\LONG{} \LONG{} \DOWN{}]{} \hfill \break
\noindent
Moves the cursor to the conclusion of the last subgoal.
\end {description}

\section{Motion within the Proof Tree}

{} The following commands allow one to move around within the proof.  They
change the current node by moving into or out of subproofs.  It is helpful,
in memorizing these commands, to think of the proof tree as lying on its
side, having children of a node layed out vertically with the first child
level with its parent.
%
\begin {description}
\item [\RIGHT{}] \hfill \break
\noindent
    Only has an effect when the cursor is in a subgoal;
    the node for that subgoal becomes the current node.

\item [\LEFT{}] \hfill \break
\noindent
    Sets the current node to be the parent of the current node.
    The cursor will still point to the current goal, but the current goal
    will be displayed as one of the subgoals of the parent.

\item [\DIAG{}] \hfill \break
    If the cursor is not in the goal \DIAG{} moves it to point to
    the goal.  If it {\em is} in the goal, the parent of the current node will
    become the new current node, (as in \LEFT{}) and the cursor will point to
    its goal (unlike \LEFT{}).

\item[\LONG{} \DIAG{}]{} \hfill \break
\noindent
    Has the same effect as four \DIAG{}s.

\item[\LONG{} \LONG{} \DIAG{}]{} \hfill \break
\noindent
    Sets the current node to be the root of the proof.
    The cursor will point to the environment of its goal.

\item[\LONG{} \RIGHT{}]{} \hfill \break
    Sets the current node to be the next unproved node of the proof tree.
    The search is done in preorder and wraps around the tree if necessary.
    If the proof has no incomplete nodes the current node is set to be the
    root.
\end {description}
%
Any other combination of \LONG{} and the arrow keys is ignored.
Extra \LONG{}s are ignored.

The cursor may also be moved within a single window by pointing at the
desired location and clicking mouse-\JUMP.  However, note that scrolling
the contents of the window cannot be done just with the mouse.  Pointing at
a non-active window and clicking mouse-\JUMP\ will make the window active
(but leave the cursor where it was when the window was last active).


\section{Initiating a Refinement}

To refine a goal, a refinement rule must be supplied.
To do this, first move the cursor to point to the rule of the current node
and press \SEL{};
this will bring up an instance of the text editor in a window labeled
{\tt EDIT rule of {\it thmname}}.
Type the rule into this window and press \EXIT.
\prl{} will parse the rule and try to apply it to the goal of the current proof
tree node.
If it succeeds the proof window will be redrawn with new statuses and
subgoals as necessary.
If it fails then one of two things may happen.
If the error is severe, the status of the node (and the proof) will be set to
{\em bad}, an error message will appear in the command/status window,
and the rule will be set to {\tt ??bad refinement rule??}.
If the error is mild and due to a missing input,
\prl{} will display a {\tt HELP} message in the command/status window
and leave the rule window on the screen so that it can be fixed.

If an existing rule is selected for editing \prl{} will copy it to the
edit window.  If the text of the rule is changed, then when \EXIT\ is
pressed \prl{} will again parse and refine the current goal.  When it
does it will throw away any proofs for the children of the current
node, even if they could have been applied to the new children of the
current node.  On the other hand, if the text is has not been touched,
then when \EXIT\ is pressed \prl{} will not reparse the rule.

{\em Warning.} If the text has been touched at all, then, even if the
text is {\em identical} to when the window was opened, the text will be
reparsed and the entire subproof will be lost.

\section{Showing Goals, Subgoals and Rules}

{}In some circumstances it is convenient to be able to
copy goals or subgoals into other proofs.
This cannot be done directly, but there is a roundabout way.
If \SEL{} is pressed while the cursor is in a goal Nuprl creates a read-only
text window containing the text of the goal.
The window will be labeled {\tt SHOW goal of {\it thmname}}.
The copy mechanism of the text editor can then be used to copy
portions of the goal to other windows.


\section{Invoking Transformation Tactics}

Transformation tactics are tactics which act as proof transformers
rather than as generalized refinement rules.
To invoke a transformation{} tactic,
press \TRANSFORM.
This will bring up a text edit window labelled
{\tt Transformation Tactic}.
Type the name of the tactic and any arguments into the window and press
\EXIT;
Nuprl will apply the tactic and redisplay the proof window to show any
effects.
If the tactic returns an error message it will be displayed in the
command/status window.

To apply a transformation tactic, the system proceed as with the
tactic rule (see Section~\ref{tactic-rule}), with the following two
differences.  First, the argument given to the tactic, instead of
being a degenerate proof, is the entire proof tree $p$ rooted at the
node to which the tactic is applied.  Second, the editor replaces $p$
with the proof resulting from the application of the tactic.



\section{The Autotactic}
Nuprl provides the facility for
a distinguished transformation tactic known as the {\em autotactic}.
This tactic is invoked automatically on the results of each primitive
refinement step.  After each primitive
refinement performed in the refinement editor
the autotactic is run on the subgoals as a transformation
tactic.  There is a default autotactic, but any refinement or
transformation tactic may be used.  For example, to set the autotactic to
a transformation tactic called {\tt transform\_tac}, the following ML code
is used:
\begin{verbatim}
        set_auto_tactic `transform_tac`;;
\end{verbatim}
This code could be used either in ML mode or placed
in an ML object (and checked).  To see what the current autotactic is,
type 
\begin {verbatim}
        show_auto_tactic ();;
\end{verbatim}
in ML mode.

The autotactic is applied to subgoals of primitive refinements only when a
proof is being edited.  When performing a library load or when using a
tactic while editing a proof, the autotactic is inhibited.


