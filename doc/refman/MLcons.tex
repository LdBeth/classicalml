\chapter{Basic ML Constructors and Destructors}

\label{MLExts}

In what follows, all universe levels must be strictly positive.  Unless
otherwise stated the tokens {\tt `nil`} and {\tt `NIL`} may be used as
identifiers to indicate that no identifier should label the new hypothesis.


\section{Rule Constructors}
The constructors are categorized
in the same manner as the rules.
Each constructor is presented in the form
\mbox{\tt constructor\_name(parameter names): type} where \tid{type} is the
type of the constructor.  The parameter names are given only to indicate the
correspondence between the constructor arguments and the components of the
rule.
Following are some conventions regarding the parameter names we use.
\begin{itemize}
\item{\tt hyp:int} is used to indicate a hypothesis.
\item{\tt where:int} is used to indicate which equand of an equality should be
reduced in a computation rule.
\item{\tt level:int} is used to give a universe level.
\item{\tt using:term} is used to indicate a using term.
\item{\tt over\_id:tok, over:term} are used to indicate an over term.
If the {\tt over\_id} is the token {\tt `nil`}, the over term is ignored.
\end{itemize}
All parameters corresponding to new identifiers have the same names as the
new names given in the corresponding rule, and should be ML tokens.
To specify the default identifier for an optional new identifier, give the
token {\tt `nil`}.

Unlike the rules, each constructor needs a unique name.
The general form of the names is {\em base-type}\_{\em kind}\_{\em
qualifier}, where {\em base-type} is a prl type and {\em kind} is one of
{\tt intro}, {\tt elim}, {\tt equality}, {\tt formation} or {\tt
computation}.
So, for example, {\tt list\_equality\_nil} is the name of the constructor for
the rule which specifies how to show two {\tt nil} terms are the same in a
{\tt list} type, and {\tt int\_elim} is the name of the constructor for the elim
rule for integers.
There are some exceptions to this general pattern, the main one being that
the name of the constructor for the rule for showing two canonical type
terms to be equal in some universe is {\em type}{\tt \_equality} rather than
{\tt universe\_equality\_}{\em type}.


\subsection*{Atom}
\subsubsection*{formation}
\uglymanentry{universe\_intro\_atom: rule}

\uglymanentry{atom\_equality: rule}

\subsubsection*{canonical}
\uglymanentry{atom\_intro(token): term -> rule}

\uglymanentry{atom\_equality\_token: rule}

\subsubsection*{equality}
\uglymanentry{atom\_eq\_equality: rule}

\subsubsection*{computation}
\uglymanentry{atom\_eq\_computation(where): int -> rule}



\subsection*{Void}
\subsubsection*{formation}
\uglymanentry{universe\_intro\_void: rule}

\uglymanentry{void\_equality: rule}

\subsubsection*{noncanonical}
\uglymanentry{void\_elim(hyp): int -> rule}

\uglymanentry{void\_equality\_any: rule}



\subsection*{Int}
\subsubsection*{formation}
\uglymanentry{universe\_intro\_integer: rule}

\uglymanentry{int\_equality: rule}

\subsubsection*{canonical}
\uglymanentry{integer\_intro\_integer(c): int -> rule}

\uglymanentry{integer\_equality\_natural\_number: rule}

\subsubsection*{noncanonical}
\uglymanentry{integer\_intro\_addition: rule}

\uglymanentry{integer\_intro\_subtraction: rule}

\uglymanentry{integer\_intro\_multiplication: rule}

\uglymanentry{integer\_intro\_division: rule}

\uglymanentry{integer\_intro\_modulo: rule}

\uglymanentry{integer\_equality\_minus: rule}

\uglymanentry{integer\_equality\_addition: rule}

\uglymanentry{integer\_equality\_subtraction: rule}

\uglymanentry{integer\_equality\_multiplication: rule}

\uglymanentry{integer\_equality\_division: rule}

\uglymanentry{integer\_equality\_modulo: rule}

\uglymanentry{integer\_elim(hyp y z): int -> tok -> tok -> rule}

\uglymanentry{integer\_equality\_induction(over\_id over u v):
tok -> term -> tok -> tok -> rule}

\subsubsection*{equality}
\uglymanentry{int\_eq\_equality: rule}

\uglymanentry{int\_less\_equality: rule}


\subsubsection*{computation}
\uglymanentry{integer\_computation(kind where): tok -> int -> rule}
where kind should be one of `UP`, `BASE` or `DOWN`.

\uglymanentry{int\_eq\_computation(where): int -> rule}

\uglymanentry{int\_less\_computation(where): int -> rule}



\subsection*{Less}
\subsubsection*{formation}
\uglymanentry{universe\_intro\_less: rule}

\uglymanentry{less\_equality: rule}

\subsubsection*{equality}
\uglymanentry{axiom\_equality\_less: rule}



\subsection*{List}
\subsubsection*{formation}
\uglymanentry{universe\_intro\_list: rule}

\uglymanentry{list\_equality(level): int -> rule}

\subsubsection*{canonical}
\uglymanentry{list\_intro\_nil(level): int -> rule}

\uglymanentry{list\_equality\_nil(level): int -> rule}  


\uglymanentry{list\_intro\_cons: rule}

\uglymanentry{list\_equality\_cons: rule}

\subsubsection*{noncanonical}
\uglymanentry{list\_elim (hyp u v w):
   int -> tok -> tok -> tok -> rule}

\uglymanentry{list\_equality\_induction (over\_id over using u v w): 
tok -> term -> term -> tok -> tok -> tok -> rule}

\subsubsection*{computation}
\uglymanentry{list\_computation(where): int -> rule}



\subsection*{Union}
\subsubsection*{formation}
\uglymanentry{universe\_intro\_union: rule}

\uglymanentry{union\_equality: rule}

\subsubsection*{canonical}
\uglymanentry{union\_intro\_left(level): int -> rule}

\uglymanentry{union\_equality\_inl(level): int -> rule}


\uglymanentry{union\_intro\_right(level): int -> rule}

\uglymanentry{union\_equality\_inr(level): int -> rule}

\subsubsection*{noncanonical}
\uglymanentry{union\_elim(hyp x y): int -> tok -> tok -> rule}

\uglymanentry{union\_equality\_decide(over\_id over using u v):
tok -> term -> term -> tok -> tok -> rule}

\subsubsection*{computation}
\uglymanentry{union\_computation(where): int -> rule}



\subsection*{Function}
\subsubsection*{formation}
\uglymanentry{universe\_intro\_function(x term): tok -> term -> rule}

\uglymanentry{function\_equality(y): tok -> rule}

\uglymanentry{universe\_intro\_function\_independent: rule}

\uglymanentry{function\_equality\_independent: rule}

\subsubsection*{canonical}
\uglymanentry{function\_intro(level y): int -> tok -> rule}

\uglymanentry{function\_equality\_lambda(level z): int -> tok -> rule}

\subsubsection*{noncanonical}
\uglymanentry{function\_elim(hyp on y): int -> term -> tok
-> rule}

\uglymanentry{function\_elim\_independent(hyp y): int -> tok -> rule}

\uglymanentry{function\_equality\_apply(using):  term -> rule}

\subsubsection*{equality}
\uglymanentry{extensionality(y): tok -> rule}

\subsubsection*{computation}
\uglymanentry{function\_computation(where): int -> rule}



\subsection*{Product}
\subsubsection*{formation}
\uglymanentry{universe\_intro\_product(x left\_term): tok -> term -> rule}

\uglymanentry{product\_equality(y): tok -> rule}

\uglymanentry{universe\_intro\_product\_independent: rule}

\uglymanentry{product\_equality\_independent: rule}

\subsubsection*{canonical}
\uglymanentry{product\_intro\_dependent(left\_term level y): term -> int -> 
tok -> rule}

\uglymanentry{product\_equality\_pair(level y): int -> tok -> rule}

\uglymanentry{product\_intro: rule}

\uglymanentry{product\_equality\_pair\_independent: rule}

\subsubsection*{noncanonical}
\uglymanentry{product\_elim(hyp u v): int -> tok -> tok -> rule}

\uglymanentry{product\_equality\_spread(over\_id over using u v):
tok -> term -> term -> tok -> tok -> rule}

\subsubsection*{computation}
\uglymanentry{product\_computation(where): int -> rule}



\subsection*{Quotient}
\subsubsection*{formation}
\uglymanentry{universe\_intro\_quotient(A E x y z): term -> 
term -> tok -> tok -> tok -> rule}

\uglymanentry{quotient\_formation(x y z): tok -> tok -> tok -> rule}


\subsubsection*{canonical}
\uglymanentry{quotient\_intro(level): int -> rule}

\uglymanentry{quotient\_equality\_element(level): int -> rule}

\subsubsection*{noncanonical}
\uglymanentry{quotient\_elim (hyp level v w): int -> int -> tok -> 
tok -> rule}

\subsubsection*{equality}
\uglymanentry{quotient\_equality(r s): tok  -> tok -> \ rule} 

\uglymanentry{quotient\_equality\_element(level): int  -> rule}



\subsection*{Set}
\subsubsection*{formation}
\uglymanentry{universe\_intro\_set(x type): tok -> term -> rule}

\uglymanentry{set\_formation(y): tok -> rule}

\uglymanentry{universe\_intro\_set\_independent: rule}

\uglymanentry{set\_formation\_independent: rule}

\subsubsection*{canonical}
\uglymanentry{set\_intro(level left\_term y): int -> term -> tok -> rule}

\uglymanentry{set\_equality\_element(level y): int -> tok -> rule}

\uglymanentry{set\_intro\_independent: rule} 


\uglymanentry{set\_equality\_element: rule}

\subsubsection*{noncanonical}
\uglymanentry{set\_elim(hyp level y): int -> int -> tok -> rule}

\subsubsection*{equality}
\uglymanentry{set\_equality(z): tok -> rule}



\subsection*{Equality}
\subsubsection*{formation}
\uglymanentry{universe\_intro\_equality(type number\_terms): term -> int -> rule}

\uglymanentry{equal\_equality: rule}

\subsubsection*{canonical}
\uglymanentry{axiom\_equality\_equal: rule}

\uglymanentry{equal\_equality\_variable: rule}



\subsection*{Universe}
\subsubsection*{canonical}
\uglymanentry{universe\_intro\_universe (level): int -> rule}

\uglymanentry{universe\_equality: rule}

\subsection*{Tactic}
\uglymanentry{tactic(text): tok -> rule}


\subsection*{Miscellaneous}
\subsubsection*{hypothesis}
\uglymanentry{hyp(hyp): int -> rule}

\subsubsection*{thinning}
\uglymanentry{thinning(hyp\_num\_list): int list \map rule}

\subsubsection*{sequence}
\uglymanentry{seq(term\_list id\_list): (term list) -> (tok list) -> rule}
The length of the token list should match the length of the term list.

\subsubsection*{lemma}
\uglymanentry{lemma(lemma\_name x): tok -> tok -> rule}

\subsubsection*{def}
\uglymanentry{def(theorem x): tok -> tok -> rule}

\subsubsection*{explicit intro}
\uglymanentry{explicit\_intro(term): term -> rule} 

\subsubsection*{cumulativity}
\uglymanentry{cumulativity(level): int -> rule}

\subsubsection*{substitution}
\uglymanentry{substitution(level equality\_term over\_id over):
int -> term -> tok -> term -> rule}

\subsubsection*{direct computation}
\uglymanentry{direct\_computation(tagged\_term): term -> rule}

\uglymanentry{direct\_computation\_hyp(hyp tagged\_term):
int -> term -> rule}

\uglymanentry{reverse\_direct\_computation(tagged\_term): term -> rule}

\uglymanentry{reverse\_direct\_computation\_hyp(hyp tagged\_term):
int -> term -> rule}

\subsubsection*{equality}
\uglymanentry{equality: rule}

\subsubsection*{arith}
\uglymanentry{arith(level): int -> rule}

\subsubsection*{monotonicity}
\uglymanentry{monotonicity(op hyp1 hyp2): tok -> int -> int -> rule}

\subsubsection*{division}
\uglymanentry{division: rule}

\section{Term Destructors}

Below are the term destructors and their types.

\uglymanentry{destruct\_universe: term -> int} 
The integer is the universe
level.

\uglymanentry{destruct\_any: term -> term} 


\uglymanentry{destruct\_token: term -> tok} 


\uglymanentry{destruct\_natural\_number: term -> int} 


\uglymanentry{destruct\_minus: term -> term} 


\uglymanentry{destruct\_addition: term -> (term \#\ term)}
 The results are
the left and right terms.

\uglymanentry{destruct\_subtraction: term -> (term \#\ term)} 


\uglymanentry{destruct\_multiplication: term -> (term \#\ term)} 


\uglymanentry{destruct\_division: term -> (term \#\ term)} 


\uglymanentry{destruct\_modulo: term -> (term \#\ term)} 


\uglymanentry{destruct\_integer\_induction: term -> (term \#\ term \#\ term \#\
term)}
The results are the value, down term, base term and up term.

\uglymanentry{destruct\_list: term -> term} 
 Returns the type of the list.

\uglymanentry{destruct\_cons: term -> (term \#\ term)} 
 Returns the head and
tail.

\uglymanentry{destruct\_list\_induction: term -> (term \#\ term \#\ term)}
Returns the value, base term and up term.

\uglymanentry{destruct\_union: term -> (term \#\ term)}
 Results in the left
and right types.

\uglymanentry{destruct\_inl: term -> term} 


\uglymanentry{destruct\_inr: term -> term} 


\uglymanentry{destruct\_decide: term -> (term \#\ term \#\ term)}
  The result
is the value, the left term and the right term.

\uglymanentry{destruct\_product: term -> (tok \#\  term \#\ term)}
 The result
is the bound identifier, the left term and the right term.

\uglymanentry{destruct\_pair: term -> (term \#\ term)}
 Returns the left term
and the right term.

\uglymanentry{destruct\_spread: term -> (term \#\ term)}
 Returns the value
and the term.

\uglymanentry{destruct\_function: term -> (tok \#\ (term \#\ term))}
 The
result is the bound identifier, the left term and the right term.

\uglymanentry{destruct\_lambda: term -> (tok \#\ term)}
  Returns the
bound identifier and the term.

\uglymanentry{destruct\_apply: term -> (term \#\ term)}
  Returns the
function and the argument.

\uglymanentry{destruct\_quotient: term -> (tok \#\ (tok \#\ (term \#\ term)))}
Returns the two bound identifiers, the left type and the right type.

\uglymanentry{destruct\_set: term -> (tok \#\ (term \#\ term))}
  Returns the
bound identifier, the left type and the right type.

\uglymanentry{destruct\_var: term -> tok} 


\uglymanentry{destruct\_equal: term -> ((term list) \#\ term)}
 Returns a list
of the equal terms and the type of the equality.

\uglymanentry{destruct\_less: term -> (term \#\ term))}
 Returns the
left term and the right term.

\uglymanentry{destruct\_bound\_id: term -> ((tok list) \#\ term)}
 Returns a
list of the bound identifiers and the type.

\uglymanentry{destruct\_term\_of\_theorem: term -> tok}
 Returns the name of the
theorem.

\uglymanentry{destruct\_atomeq: term -> (term \#\ (term \#\ (term \#\ term)))}
Returns the left term, right term, if term and else term.

\uglymanentry{destruct\_inteq: term -> (term \#\ (term \#\ (term \#\ term)))}
Returns the left term, right term, if term and else term.

\uglymanentry{destruct\_intless: term -> (term \#\ (term \#\ (term \#\ term)))}
Returns the left term, right term, if term and else term.

\uglymanentry{destruct\_tagged: term -> (int \#\ term)}
The integer returned is zero if the tag is {\tt *}.



\section{Term Constructors}
For each term type there is a corresponding constructor that is the
curried inverse to the destructor given above.  For example,
\begin{quote}
{\tt make\_inteq\_term: term -> term -> term -> term -> term}.
\end{quote}








