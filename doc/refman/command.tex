\chapter{The Command Language}

With the {\em command language} one can
create, delete and manipulate library objects, evaluate terms, call up the
editors, and ``check'' (i.e., parse and generate code for) objects.
{}When a ``{\tt P>}''{} prompt appears in the command/status window,
the command processor is waiting for a command.%
{}%
\footnote{If the cursor is within an edit window \CMD{} must
be pressed to get the ``{\tt P>}'' prompt.}
Commands may be abbreviated to their shortest unique prefixes; for example,
{\tt check} can be abbreviated to {\tt ch}, and {\tt create} can be
abbreviated to {\tt cr}.  Names of library objects given as arguments to
commands may also be abbreviated: the abbreviation refers to the object
with exactly the abbreviated name if such an object exists, or else to the
first library entry whose name extends the abbreviation.  Typing errors
are corrected with the
\DEL{}{} key (to delete a character) and the \ERASE{}{} key to delete the
whole line.

The commands are divided into four groups:
a group consisting of commands that control the library window,
a group consisting of commands that manipulate objects,
a group consisting of commands that allow work to be saved between sessions,
and a miscellaneous group.

The following placeholders are used in the descriptions below
of command syntax.

\begin {description}
\item[$object$]{} is the name of an object in the library or one of the
    reserved names {\tt first} and {\tt last}

\item[$objects$]{} is $object$ or $object${\tt -}$object$.
    This provides a convenient way to refer to a range of entries.

\item[$place$]{} is {\tt top}, {\tt bot}, {\tt before} $object$ or
    {\tt after} $object$.
    ({\tt Top} refers to the slot immediately before the first entry,
    and {\tt bottom} refers to the slot immediately after the last entry.)

\item[$kind$]{} is {\tt thm}, {\tt def}, {\tt eval} or {\tt ml}.

\item[{$filename$}] is any legal file name.
  
\end {description}

\noindent
Optional arguments and keywords are indicated by curly braces.

\section{Library Display}


The library is a list of the objects defined by the user
and may contain
the theorems, definitions, evaluation bindings and ML code.
Information about the library objects is displayed in the
{\em library{} window\/}.
The {\tt jump} and {\tt scroll} commands change what is being
displayed in the library window, while
the {\tt move} command changes the order of library objects.

\begin {description}
\item[{\tt jump} $object$]{} \hfill \break \noindent {}
    The library is redisplayed so that the specified entry is visible.

\item[{\tt move} $objects$ $place$]{} \hfill \break \noindent
    {}
    The specified library objects are moved to the position just after the
    specified place.

\item[{\tt scroll} \(\{{\tt up}\}\) \(\{number\}\)]{} \hfill \break
    \noindent {}
    The library window is moved the specified number of entries, in the
    specified direction.
    The default direction is down, and the default number of entries is one.
    For example, {\tt scroll 5} shifts the window down five entries, while
    {\tt scroll up} shifts the window up one entry.

\item[{\tt up} \(\{number\}\)]{} \hfill \break \noindent {}
Scroll the library up {\em number} pages (default 1).

\item[{\tt down} \(\{number\}\)]{} \hfill \break \noindent {}
Scroll the library down {\em number} pages (default 1).

\item[{\tt top}]{} \hfill \break \noindent {}
Scroll the library to the top.  This is the same as {\tt jump first}
(when the library is non-empty).

\item[{\tt bottom}]{} \hfill \break \noindent {}
Scroll the library to the bottom.  This is similar to {\tt jump last}
(when the library is non-empty).

\end {description}

\section{Manipulating Objects}

These are the commands that directly manipulate{}
objects.
The most used of these commands is {\tt view},
since most interaction with the system takes place within the editors
started up by the {\tt view} command.

\begin {description}
\item[{\tt check} $objects$]{} \hfill \break
\noindent {}{}
    Each of the indicated objects is checked.
    If necessary the library window is redisplayed to show the 
    new statuses.

    Checking a definition object causes the text in the object 
    to be parsed.  If it is well-formed, then the definition
    becomes available for use in other objects.

    Checking an ML object simply calls ML on the text contained
    in the object.
 
    Checking an eval object causes the contained text to be
    processed as if it were entered in eval mode (see below).

    Checking a large theorem can take a while.
    If the body of the theorem was dumped to a file and has not been fully
    reloaded (see the {\tt dump} and {\tt load} commands below) then checking
    the theorem reconstructs the body of the proof.
    If it has already been reconstructed, then the proof is not changed by
    checking.

    Checking a theorem causes extraction to be done on the proof (unless
    the proof has not been touched since the last time extraction was
    performed).
    

\item[{\tt create} $name$ $kind$ $place$]{} \hfill \break
\noindent {}
    A new library entry
    of the specified name and kind is created at the specified place.
    The library is redisplayed so that the newly created object is the first
    object in the library window.

\item[{\tt delete} $objects$]{} \hfill \break
\noindent {}
    The specified objects are removed from the library.


\item[{\tt view} {\it object}]{} \hfill \break
\noindent {}
    The object is displayed in a new window.
    If the object is not already being viewed the new view will be fully
    editable; otherwise, it and all other views of the object will be made
    read-only.
    The header line of the view will say {\tt SHOW} for a read-only view
    and {\tt EDIT} for an editable view.

    The editor used depends on the kind of object.
    The refinement editor ({\em red}) is used on theorems, while the 
    text{} editor ({\em ted}) is used for other objects.
    For more information on ted and red see sections 7.4 and 7.5, respectively.
    Typing \EXIT\ in an edit window will make Nuprl delete the
    window and end its editing of the object.
    If there are other edit views on the screen the cursor will be placed in one of
    them; otherwise, the cursor will go back to the command window.

    When {\tt view} is done on a large theorem, it may take some time to
    activate red.
    If the body of the theorem has been dumped but not fully reloaded
    (see the {\tt dump} and {\tt load} commands below) then Nuprl reads the
    body in and reconstructs it before starting the editor.

\end {description}

\section{Storing Results}

The {\tt load} and {\tt dump} commands let one save results between Nuprl
sessions.  When a library is dumped to a file, proofs are given a more
compact representation.  The usual representation of a proof is a tree of
sequents and rules.  The dumped representation is a ``rule-body tree'', which
is obtained, roughly, by removing all the sequents from the usual
representation, and by converting all the rules to their printed
representations.  Thus the dumped representation contains essentially just
the text that a user would type to reconstruct the proof.  One of the 
savings that this provides has to do with tactics.  A field of the tactic
rule contains the (hidden) subproof that was built by the tactic; this
subproof is discarded in a dump, and only the text of the tactic is
retained as a record of the inference.

When libraries are loaded, the loaded proofs retain their compact
representation.  When a proof is needed, as when it is to be viewed,
checked, or extracted from, then the system will reconstruct the usual
proof tree.  This has several drawbacks.  First, it can be time
consuming: all the tactics used to construct the proof must be
re-executed.  Secondly, if the tactics that were used to construct the
proofs have since been modified, the reconstruction may fail.  Proofs
are rather rigid objects, and seemingly insignificant changes to
tactics can cause quite a bit of damage to previously constructed
proofs.  

When a theorem is extracted from, the extraction is stored with the
theorem in the library.  When a theorem is dumped to a file, if it has
an extraction, then an unparsed version of the extraction is also
dumped.  This feature can considerably reduce the need for proof
expansion.


{}{}
%
\begin {description}
\item[{\tt dump} {\it objects} \(\{{\tt to}\ {\it filename}\}\)]{} \hfill \break
\noindent {}
    Nuprl writes a representation of the selected objects from
    the current library to the file.
    If no filename is given the name of the file most recently loaded is used.
    The dumped objects are not removed from the library.

\item[{\tt load} \(\{{\tt fully}\}\) {\it place} {\tt from} {\it filename}]{} \hfill \break
\noindent{}%
    This is the command used to read in a file produced by {\tt dump}.
    The decoded library entries are added one-by-one
    to the library starting at location
    $place$.
    If an object in the dumped file has the same name as an object already in
    the library, the object in the file is ignored.
    When the load is finished a message is displayed giving the number of
    objects loaded and the number of duplicates discarded.
    The name of the loaded file is displayed in the header of the library
    window.

    Theorem objects are not fully reconstructed unless {\tt fully} is
    specified.
    The top of the proof (its goal and status) and the term extracted
from the proof are loaded, but the actual
    body of the proof (the rule applications and subgoals) is not.
    This body is loaded on demand when the theorem is checked or viewed or when
    the extractor needs the body of the proof.
    When the body is actually loaded, the theorem's status is recalculated.
    If the new status is different from the status the theorem had when it was
    dumped, a warning is given.  


\item[{\tt save} {\it name}]{} \hfill \break \noindent {} This allows
temporary storing of Nuprl sessions in the Lisp environment.  The
command will save the current Nuprl state, attaching the name {\em
name}.  By ``state'' is meant all aspects of the state of the Nuprl
session except the current window configuration; this includes the
library, the values and types associated with ML identifiers, and the
eval environment.  There is a special state name ``initial'' which is
the state of Nuprl right after it is first loaded --- in particular it
includes the default collection of tactics that is loaded with the
system (unless the loading of the collection was overridden).  This
initial state is restored automatically after the {\em save} command
is invoked.  Note that since states are stored in the Lisp
environment, they disappear when Lisp is exited.

\item[{\tt restore} {\it name}]{} \hfill \break \noindent {}
Restore a copy of the state named {\it name}, replacing the current state.

\item[{\tt copystate} {\it name} {\it new-name}]{} \hfill \break \noindent {}
Create a copy of state named {\em name}, giving it the name {\em
new-state}.

\item[{\tt states} ]{} \hfill \break \noindent {}
List the names of states that have been created (and not killed).

\item[{\tt kill} {\it name}]{} \hfill \break \noindent {}
Make the state named {\it name} go away.


\end {description}


\section{Miscellaneous Commands}

This section contains those commands which do not fit in any specific
category.
%
\begin {description}
\item[{\tt eval}]{}{}
    Puts the command window into {\em eval} mode.
    The prompt changes from ``{\tt P>}'' to ``{\tt EVAL>}''{} to
    indicate the change.
    Two kinds of inputs can be entered in eval mode:
    terms and bindings.
    Each input must be terminated by a double semicolon, ``{\tt ;;}''.
    If a term is entered Nuprl evaluates it and prints its
    value.
    Bindings have the form \({\tt let}\ {\it id}={\it term}\).  If a binding
    is entered Nuprl evaluates the term, prints its value and binds the
    value to the identifier.
    The evaluator's binding environment persists from one eval session to the
    next.
    It also includes bindings made when EVAL objects are checked.
    Eval mode is terminated by \EXIT.

\item[{\tt ml}]{}{}
Puts the command window into {\em ML} mode.  It operates as eval mode,
except that the input (terminated by ``{\tt{};;}'') is evaluated by ML and
the result is printed in the command window.

\item[{\tt exit}]{}{} Terminates the current \prl{} session, returning
control to Lisp.


\end {description}

There is one final command, the \PRINT{}{} command.  This
``snapshot'' window operation appends a printout of the visible contents of
the current window to the current ``snapshot file''.  The default for the
current snapshot file is ``snapshot.text'' in the current user's home
directory; to change the snapshot file, use the ML function
``set\_snapshot\_file''.
