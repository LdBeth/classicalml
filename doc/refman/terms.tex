\section{Terms and Evaluation}

This documents the syntax and evaluation of the terms of Nuprl's type
theory.  In giving the procedure for evaluation and specifying the
refinement rules of the theory, we will use the following notation for
substitution.  If $0\le n$, $x_1,\dots,x_n$ are variables and
$t_1,\dots,t_n$ are terms then \[ t[t_1,\dots, t_n/x_1,\dots,x_n] \]
is the result of simultaneously substituting $t_i$ for $x_i$.  

Figure \ref{TermTable} shows the terms\index{terms}{} of \nprl{}. 
Variables are terms, although since they are not closed they are not executable.
Variables are written as identifiers, with
distinct identifiers indicating distinct variables.\footnote{
An identifier is any string of letters, digits, underscores or
at--signs that starts with a letter.
The only identifiers which cannot be used for variables are
{\tt term\_of} and those which serve as operator names, such as
{\tt int} or {\tt spread}.
}
Nonnegative integers are written
in standard decimal notation.  There is
no way to write a negative\index{negative integer}{}
integer in \nprl{}; the best one can do is to write a noncanonical
term, such as {\tt -5}, which evaluates to a negative integer.
Atom constants are written as character strings enclosed in 
double quotes, with distinct strings indicating distinct atom constants.
\begin{figure}
\hrule{}
\vspace{2ex}
\tt
\newcommand{\parg}{{\tiny{$\wedge$}}}
\newcommand{\bvx}{{\small{$x$}}}
\newcommand{\bvy}{{\small{$y$}}}
\newcommand{\bvu}{{\small{$u$}}}
\newcommand{\bvv}{{\small{$v$}}}
\newcommand{\wide}[1]{\multicolumn{2}{l}{#1}}
\begin{minipage}{\hsize}
\begin{tabbing}
\hspace{1in} 	\= \hspace{1in}	\= \hspace{1in}	\=		\kill
$x$		\> $n$		\> $i$		\> void		\\[1ex]
int		\> atom		\> axiom	\> nil		\\[1ex] 
U$k$		\> inl($a$)	\> inr($a$)	\> $A$ list	\\[1ex] 
\= \= \hspace{1in}
		\= \hspace{1in}	\= \hspace{1in}	\=		\kill
\verb+\+\= $x$.\= $b$
		\> $a$<$b$	\> <$a$,$b$>	\> $a$.$b$	\\[-.6ex]
	\> \bvx	\> \bvx						\\[1ex] 
\hspace{1in} 	\= \= \hspace{1in}
				\= \hspace{1in}	\= \= 
							\kill
$A$\#$B$	\> $x$:$A$\#\= $B$
				\> $A$->$B$	\> $x$:$A$->\= $B$
								\\[-.6ex]
		\> \bvx	    \> \bvx
				\>		\> \bvx	    \> \bvx \\[1ex] 
\hspace{1in} 	\= \hspace{1in}
				\= \= \= \hspace{1in}	\= 
								\kill
$A$|$B$ \> $A$//$B$	\> $x$,\= $y$:$A$//\= $B$
				\> \{$A$|$B$\}									\\[-.6ex]
	\> \> \bvx
		       \> \bvy     \> \bvx  \\[-1.5ex]
	\>	\>	\>	   \> \bvy  \\[1ex]
\hspace{1in} 	\= \= \= \hspace{1in}
				\= \hspace{1in}	\= \= \=
								\kill
	 \{\= $x$:$A$|\= $B$\} \> $a$ = $b$ in $A$\\[-.6ex]
 \> \bvx    \> \bvx \\[1ex]

{\scriptsize{canonical if closed}}\\[-4pt]
\makebox[\hsize]{\dotfill{}}\\[-1pt]
\scriptsize{noncanonical if closed}
\end{tabbing}
\begin{tabbing}
\= \hspace{1in}	\= \hspace{1in}	\= \hspace{1in}	\= \=			\kill
-\= $a$		\> any($a$)	\> $t$($a$)	\> $a$+\= $b$	       \\[-1ex]
 \> \parg	\>		\> \parg	\> \parg
						       \> \parg		\\[1ex]
\= \hspace{1in}	\= \= \hspace{1in}
				\= \= \hspace{1in}
						\= \=			\kill
$a$-\= $b$	\> $a$*\= $b$	\> $a$/\= $b$	\> $a$ mod \= $b$      \\[-1ex]
\parg
    \> \parg	 \> \parg
		        \> \parg
				\> \parg
				       \> \parg \> \parg   \> \parg    \\[1ex]
\= \= \= \= \hspace{2in}	\= \= \= \= \= \= 			\kill
spread(\= $a$;\= $x$,\= $y$.\= $t$)
				\> decide(\= $a$;\= $x$.\= $s$;\= $y$.\= $t$) \\[-.6ex]
       \> \parg
	      \> \bvx
		     \> \bvy
			    \> \bvx
				\> 	  \> \parg
						 \> \bvx
							\> \bvx \> \bvy
								      \> \bvy \\[-1.5ex]
	\>    \>     \>	    \> \bvy			
\end{tabbing}
\begin{tabbing}
\= \= \= \= \= \hspace{2in}	\= \= \= \= \= \= \= \= 		\kill
list\_ind(\= $a$;$s$;\= $x$,\= $y$,\= $u$.\= $t$)
\> ind(\= $a$;\= $x$,\= $y$.\= $s$;$b$;\= $u$,\= $v$.\= $t$) \\[-.6ex]
	  \> \parg   \> \bvx
			    \> \bvy
				   \> \bvu
					  \> \bvx
\>     \> \parg
	      \> \bvx
		     \> \bvy
			    \> \bvx    \> \bvu
					      \> \bvv
						     \> \bvu \\[-1.5ex]
	  \>	     \>	   \>	   \>	  \> \bvy
\>	\>    \>     \>	    \> \bvy    \>      \>    \> \bvv \\[-1.2ex]
	  \>	     \>	   \>	   \>	  \> \bvu	     \\[1ex]
\= \= \hspace{2in}		\= \= \= \hspace{2in}		\= \= \= \kill
atom\_eq(\= $a$;\= $b$;$s$;$t$) \> int\_eq(\= $a$;\= $b$;$s$;$t$)
																\\[-1ex]
	 \> \parg
		\> \parg	\>	   \> \parg
						  \> \parg	\>	\\[1ex]
									       
	 less(\= $a$;\= $b$;$s$;$t$)\\[-.6ex]
	 \> \parg \> \parg
\end{tabbing}
\vspace{1ex}
\mbox{\small{
\begin{tabular}{ll}
	$x,y,u,v$	& range over variables. \\
	$a,b,s,t,A,B$	& range over terms.  \\
	$n$		& ranges over integers. \\
	$k$		& ranges over positive integers. \\
	$i$		& ranges over atom constants. 
\end{tabular}
 } }
 
\vspace{2pt}
{\small

	Variables written below a term indicate 
	where the variables become bound.\\
	``\parg{}'' indicates principal arguments.\\
} 
\label{TermTable}
\caption{Terms}
\vspace{2pt}
\hrule{}
\end{minipage}
\end{figure}

The free\index{free variables}{} occurrences of a variable $x$
in a term $t$ are the occurrences
of $x$ which either {\sl are} $t$ or are free in the immediate subterms of $t$,
excepting those occurrences of $x$ which become {\sl bound}\index{bound variables}{}
in $t$.  In figure~\ref{TermTable}
the variables written below the terms indicate which variable
occurrences become bound; some examples are explained below.
\begin{itemize}
\item	In {\tt $x$:$A$\#$B$} the $x$ in front of the colon becomes bound
	and any free occurrences of $x$ in $B$ become bound.
	The free occurrences of variables in {\tt $x$:$A$\#$B$} are
	all the free occurrences of variables in $A$ and
	all the free occurrences of variables in $B$
	except for $x$.
\item	In {\tt <$a$,$b$>} no variable occurrences become bound; hence,
	the free occurrences of variables in {\tt <$a$,$b$>} are
	those of $a$ and those of $b$.
\item	In {\tt spread($s$;$x$,$y$.$t$)} the $x$ and $y$ in front of the dot
	and any free occurrences of $x$ or $y$ in $t$ become bound.
\end{itemize}


Parentheses may be used freely around terms and often must be used to resolve
ambiguous notations correctly.  Figure~\ref{ParseTable} gives the
relative\index{relative precedence}{} precedences and
associativities\index{associativity}{}
of \nprl{} operators.

\begin{figure}
\hrule{}
\vspace{2ex}
\begin{center}
\begin{tabular}{ll}
\multicolumn{2}{c}{Lower Precedence}\\[1ex]
\verb'=',\verb'in'            & left associative \\[1ex]
\verb'#',\verb'->',\verb'|',\verb'//'      & right associative \\[1ex]
\verb'<' (as in {\tt $a$<$b$})             & left associative \\[1ex]
\verb'+',\verb'-' (infix) & left associative \\[1ex]
\verb'*',\verb'/',\verb'mod'         & left associative \\[1ex]
\verb'inl',\verb'inr',\verb'-' (prefix) & ---\\[1ex]
\verb'.' (as in {\tt $a$.$b$})	& right associative \\[1ex]
\verb'\x.' 		& ---  \\[1ex]
\verb'list'            & --- \\[1ex]
{\tt ($a$)} (as in {\tt $t$($a$)}) & --- \\[1ex]
\multicolumn{2}{c}{Higher Precedence}
\end{tabular}
\end{center}
\caption{Operator Precedence}
\label{ParseTable}
\vspace{2pt}
\hrule{}
\end{figure}
\index{operator precedence}{}

The closed terms above the dotted line in figure~\ref{TermTable} are the 
canonical\index{canonical terms}{} terms, while the closed terms below it
are the noncanonical\index{noncanonical terms}{} terms.  Note that
carets appear below most of the noncanonical forms; these indicate the
{\sl principal\index{principal argument}{} argument places} of those
terms.  This notion is used in the 
evaluation procedure below.  Certain terms are
designated as {\sl redices}\index{redex}{}, and each redex has a
unique {\sl contractum}\index{contractum}{}.
Figure~\ref{ReductionTable} shows all redices and their contracta.

\begin{figure}[tp]
\hrule{}
\begin{center}
\begin{tabular}{lp{2.5in}}
{\bf Redex}				& {\bf Contractum} \\[4pt]
{\tt (\verb+\+$x$.$b$)($a$)}		& $b[a/x]$ \\[2pt]
{\tt spread(<$a$,$b$>;$x$,$y$.$t$)}	& $t[a,b/x,y]$ \\[2pt]
{\tt decide(inl ($a$) ;$x$.$s$;$y$.$t$)}	& $s[a/x]$ \\[2pt]
{\tt decide(inr ($b$) ;$x$.$s$;$y$.$t$)}	& $t[b/y]$ \\[2pt]
{\tt list\_ind(nil;$s$;$x$,$y$,$u$.$t$)} & $s$ \\[2pt]
{\tt list\_ind($a$.$b$;$s$;$x$,$y$,$u$.$t$)} & $t[a,b,\mbox{\tt list\_ind($b$;$s$;$x$,$y$,$u$.$t$)}/x,y,u]$ \\[2pt]
{\tt atom\_eq($i$;$j$;$s$;$t$)}		& $s$ if $i$ is $j$; $t$ otherwise \\[2pt]
{\tt int\_eq($m$;$n$;$s$;$t$)}		& $s$ if $m$ is $n$; $t$ otherwise \\[2pt]
{\tt less($m$;$n$;$s$;$t$)}		& $s$ if $m$ is less than $n$; $t$ otherwise \\[2pt]
{\tt -$n$}				& the negation of $n$ \\[2pt]
{\tt $m$+$n$}				& the sum of $m$ and $n$ \\[2pt]
{\tt $m$-$n$}				& the difference \\[2pt]
{\tt $m$*$n$}				& the product \\[2pt]
{\tt $m$/$n$}				& {\tt 0} if $n$ is {\tt 0}; otherwise,
					  the floor of the obvious
					  rational. \\[2pt]
{\tt $m$ mod $n$}			& {\tt 0} if $n$ is {\tt 0};
					  otherwise, the positive integer nearest {\tt 0}
					  that differs from $m$ by a multiple 
					  of $n$. \\[2pt]
{\tt ind($m$;$x$,$y$.$s$;$b$;$u$,$v$.$t$)} & $b$ if $m$ is {\tt 0}; \\[-1\parskip]
					& $t[m,${\tt ind($m-1$;$x$,$y$.$s$;$b$;$u$,$v$.$t$)}$/u,v]$
					  if $m$ is positive;\\[-1\parskip]
					& $s[m,${\tt ind($m+1$;$x$,$y$.$s$;$b$;$u$,$v$.$t$)}$/x,y]$
					  if $m$ is negative.
\end{tabular}
\end{center}
\vspace{-4ex}
\scriptsize{
\begin{tabular}{ll}
$a, b, s, t$	& range over terms. \\
$x, y, u, v$	& range over variables. \\
$m, n$		& range over integers. \\
$i, j$		& range over atom constants.
\end{tabular} 
}
\caption{Redices and Contracta}
\vspace{2pt}
\hrule{}
\label{ReductionTable}
\end{figure}

\label{EvalProc}
The evaluation\index{evaluation}{} procedure is as follows.
Given a (closed) term $t$, 
\begin{description}{}
\item	If $t$ is canonical then the procedure terminates with result $t$.
\item	Otherwise, execute the evaluation procedure on each
	principal argument of $t$, and if each has a value,
	replace the principal arguments of $t$ by their respective
	values; call this term $s$.
\item	If $s$ is a redex then the procedure for evaluating $t$ is continued
	by evaluating the contractum of $s$.
\item	If $s$ is not a redex then the procedure is terminated
	without result; $t$ has no value.
\end{description}

