\chapter{The Library}

The library{} is a list of all the objects defined by the user
and includes
theorems, definitions, evaluation bindings and ML code.
The library{} window displays information about
the various objects---
their names, types, statuses and summary descriptions.
In this section we give a short description of the
different kinds of objects and their statuses.

\section{Object Types}

The library contains four kinds of objects: {} DEF, THM, EVAL and ML.\  DEF
objects define new notations that can be used whenever text is being
entered.  THM objects contain proofs in the form of proof trees.  Each node
of the proof tree contains a number of assumptions, a conclusion, a
refinement rule and a list of children produced by applying the refinement
rule to the assumptions and conclusion.  THM objects are checked only when
a check command is issued or they are viewed or are used as objects from which
code is extracted.

EVAL objects contain lists of bindings, where{} a binding has the form
\({\tt let}\ {\it id}={\it term}\) and is terminated by a double semicolon,
``{\tt ;;}''.  Checking an EVAL object adds its bindings to the evaluator{}
environment so that they available to the {\tt eval}{} command.  All EVAL
objects are checked when they are loaded into the library.  ML objects
contain ML programs, including tactics, which provide a general way of
combining the primitive refinement rules to form more powerful refinement
rules.  Checking an ML object enriches the ML environment.  All ML objects
are checked when they are loaded into the library.

\section{Library Entries}

Every object has associated with it a status{}{}, either {\em raw}{}, {\em
bad}{}, {\em incomplete} {} or {\em complete}{}, indicating the current
state of the object.  A {\em raw} status means an object has been changed
but not yet checked.  A {\em bad} status means an object has been checked
and found to contain errors.  An {\em incomplete} status is meaningful only
for proofs and signifies that the proof contains no errors but has not been
finished.  A {\em complete} status indicates that the object is correct and
complete.

An entry for an object in the library is displayed in the library window as
a single line contains its kind, status, type, name and a summary of the
contents.  The kind is \tid{D} for \id{definition}, \tid{t} and \tid{T} for
\id{theorem}, \tid{M} for \id{ML}, \tid{E} for \id{evaluation}.  The lower
case ``\tid{t}'' is used for theorems that have proofs in the compact
representation used for storage in files; viewing or checking such an
object will cause expansion of the proof, and this can be time consuming.
The status is encoded as a single character: {\tt ?} for raw, {\tt -} for
bad, {\tt \#} for incomplete and {\tt *} for complete.  A typical entry
might be the following.
\begin{center}
\begin{tabular}{|l|}\hline\tt
Library \\ \hline
\tt *T simple\ \ \ >> int in U1\\
\nothing\\ \hline  
\end{tabular}
\end{center}
The library window provides the mechanism for viewing these entries.
The entries are kept in a linear order, and at any one time a section of
the entries is visible in the library window.  The library placement
commands described above can be used to change the order of the
entries and to move the window around within the list of entries.

If mouse-\SEL\ is pressed when pointing at a library entry, the name of the
entry is ``typed in'' by the system.  This works in the command window and in
ted (the text editor).  So, for example, you can instantiate a definition by
\INS, mouse-\SEL\ on library object, \CR.

If mouse-\JUMP\ is pressed with the mouse above the first line in the
library window, then the library scrolls up one page.  If it is pressed
with the mouse just after the last line of the library window (but still in
the window or on the bottom border) then the library scrolls down one page.
Thus, you can scroll through the library using the mouse, instead of the
scroll commands.

If mouse-\JUMP\ is pressed with the mouse in the library scroll bar, then
the library is scrolled to the specified point.  After the scroll, the
scroll bar will indicate that the library is being shown from the point
that the mouse selected.

If mouse-\JUMP\ is pressed with the mouse on a library window line that
displays a theorem object, then the name, goal, and extracted term (if
available) will be displayed in the command window.  This allows you to see
a theorem more fully, especially if it is too long to fit on one line of
the library window.



\section{Object Dependencies and Ordering}

A correct library in Nuprl is one where every definition and theorem
refer only to objects occurring previously in the library.
Unfortunately, Nuprl does not guarantee that this property is
maintained when commands are used that modify the library.  For
example, it is possible to create a circular chain of lemma
references.

There is only one way to guarantee that a library (or sequence of
libraries) is correct.  This is to load it (them, sequentially) into
an empty library using the ``load fully'' command.  This will force a
theorem's proof to be expanded before the theorem is loaded into the
library, and so guarantee that proofs only reference theorems that
occur previously in the library.

Loading in the library into an empty library without using the
``fully'' option, and then executing ``check first-last'', will {\em
not} guarantee that the library is correct, since during the checking
of a theorem, all later theorems will be present in the library and
will retain the statuses they had when they were dumped.

Nuprl does do some dependency checking with definitions.  For example,
if a definition is deleted then the status of any entry depending on
these objects is set to BAD.  

Because of the general lack of dependency checking, a user must be
careful to keep library objects correctly ordered or reloading may
fail.  The {\tt move} command is often used to make sure that objects
occur before their uses.

